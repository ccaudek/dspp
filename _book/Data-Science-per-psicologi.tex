% Options for packages loaded elsewhere
\PassOptionsToPackage{unicode}{hyperref}
\PassOptionsToPackage{hyphens}{url}
\PassOptionsToPackage{dvipsnames,svgnames,x11names}{xcolor}
%
\documentclass[
]{memoir}
\title{Data Science per psicologi}
\author{Corrado Caudek}
\date{2021-10-10}

\usepackage{amsmath,amssymb}
\usepackage{lmodern}
\usepackage{iftex}
\ifPDFTeX
  \usepackage[T1]{fontenc}
  \usepackage[utf8]{inputenc}
  \usepackage{textcomp} % provide euro and other symbols
\else % if luatex or xetex
  \usepackage{unicode-math}
  \defaultfontfeatures{Scale=MatchLowercase}
  \defaultfontfeatures[\rmfamily]{Ligatures=TeX,Scale=1}
  \setmonofont[]{Inconsolata}
\fi
% Use upquote if available, for straight quotes in verbatim environments
\IfFileExists{upquote.sty}{\usepackage{upquote}}{}
\IfFileExists{microtype.sty}{% use microtype if available
  \usepackage[]{microtype}
  \UseMicrotypeSet[protrusion]{basicmath} % disable protrusion for tt fonts
}{}
\makeatletter
\@ifundefined{KOMAClassName}{% if non-KOMA class
  \IfFileExists{parskip.sty}{%
    \usepackage{parskip}
  }{% else
    \setlength{\parindent}{0pt}
    \setlength{\parskip}{6pt plus 2pt minus 1pt}}
}{% if KOMA class
  \KOMAoptions{parskip=half}}
\makeatother
\usepackage{xcolor}
\IfFileExists{xurl.sty}{\usepackage{xurl}}{} % add URL line breaks if available
\IfFileExists{bookmark.sty}{\usepackage{bookmark}}{\usepackage{hyperref}}
\hypersetup{
  pdftitle={Data Science per psicologi},
  pdfauthor={Corrado Caudek},
  colorlinks=true,
  linkcolor={Maroon},
  filecolor={Maroon},
  citecolor={Blue},
  urlcolor={Blue},
  pdfcreator={LaTeX via pandoc}}
\urlstyle{same} % disable monospaced font for URLs
\usepackage{color}
\usepackage{fancyvrb}
\newcommand{\VerbBar}{|}
\newcommand{\VERB}{\Verb[commandchars=\\\{\}]}
\DefineVerbatimEnvironment{Highlighting}{Verbatim}{commandchars=\\\{\}}
% Add ',fontsize=\small' for more characters per line
\usepackage{framed}
\definecolor{shadecolor}{RGB}{248,248,248}
\newenvironment{Shaded}{\begin{snugshade}}{\end{snugshade}}
\newcommand{\AlertTok}[1]{\textcolor[rgb]{0.94,0.16,0.16}{#1}}
\newcommand{\AnnotationTok}[1]{\textcolor[rgb]{0.56,0.35,0.01}{\textbf{\textit{#1}}}}
\newcommand{\AttributeTok}[1]{\textcolor[rgb]{0.77,0.63,0.00}{#1}}
\newcommand{\BaseNTok}[1]{\textcolor[rgb]{0.00,0.00,0.81}{#1}}
\newcommand{\BuiltInTok}[1]{#1}
\newcommand{\CharTok}[1]{\textcolor[rgb]{0.31,0.60,0.02}{#1}}
\newcommand{\CommentTok}[1]{\textcolor[rgb]{0.56,0.35,0.01}{\textit{#1}}}
\newcommand{\CommentVarTok}[1]{\textcolor[rgb]{0.56,0.35,0.01}{\textbf{\textit{#1}}}}
\newcommand{\ConstantTok}[1]{\textcolor[rgb]{0.00,0.00,0.00}{#1}}
\newcommand{\ControlFlowTok}[1]{\textcolor[rgb]{0.13,0.29,0.53}{\textbf{#1}}}
\newcommand{\DataTypeTok}[1]{\textcolor[rgb]{0.13,0.29,0.53}{#1}}
\newcommand{\DecValTok}[1]{\textcolor[rgb]{0.00,0.00,0.81}{#1}}
\newcommand{\DocumentationTok}[1]{\textcolor[rgb]{0.56,0.35,0.01}{\textbf{\textit{#1}}}}
\newcommand{\ErrorTok}[1]{\textcolor[rgb]{0.64,0.00,0.00}{\textbf{#1}}}
\newcommand{\ExtensionTok}[1]{#1}
\newcommand{\FloatTok}[1]{\textcolor[rgb]{0.00,0.00,0.81}{#1}}
\newcommand{\FunctionTok}[1]{\textcolor[rgb]{0.00,0.00,0.00}{#1}}
\newcommand{\ImportTok}[1]{#1}
\newcommand{\InformationTok}[1]{\textcolor[rgb]{0.56,0.35,0.01}{\textbf{\textit{#1}}}}
\newcommand{\KeywordTok}[1]{\textcolor[rgb]{0.13,0.29,0.53}{\textbf{#1}}}
\newcommand{\NormalTok}[1]{#1}
\newcommand{\OperatorTok}[1]{\textcolor[rgb]{0.81,0.36,0.00}{\textbf{#1}}}
\newcommand{\OtherTok}[1]{\textcolor[rgb]{0.56,0.35,0.01}{#1}}
\newcommand{\PreprocessorTok}[1]{\textcolor[rgb]{0.56,0.35,0.01}{\textit{#1}}}
\newcommand{\RegionMarkerTok}[1]{#1}
\newcommand{\SpecialCharTok}[1]{\textcolor[rgb]{0.00,0.00,0.00}{#1}}
\newcommand{\SpecialStringTok}[1]{\textcolor[rgb]{0.31,0.60,0.02}{#1}}
\newcommand{\StringTok}[1]{\textcolor[rgb]{0.31,0.60,0.02}{#1}}
\newcommand{\VariableTok}[1]{\textcolor[rgb]{0.00,0.00,0.00}{#1}}
\newcommand{\VerbatimStringTok}[1]{\textcolor[rgb]{0.31,0.60,0.02}{#1}}
\newcommand{\WarningTok}[1]{\textcolor[rgb]{0.56,0.35,0.01}{\textbf{\textit{#1}}}}
\usepackage{longtable,booktabs,array}
\usepackage{calc} % for calculating minipage widths
% Correct order of tables after \paragraph or \subparagraph
\usepackage{etoolbox}
\makeatletter
\patchcmd\longtable{\par}{\if@noskipsec\mbox{}\fi\par}{}{}
\makeatother
% Allow footnotes in longtable head/foot
\IfFileExists{footnotehyper.sty}{\usepackage{footnotehyper}}{\usepackage{footnote}}
\makesavenoteenv{longtable}
\usepackage{graphicx}
\makeatletter
\def\maxwidth{\ifdim\Gin@nat@width>\linewidth\linewidth\else\Gin@nat@width\fi}
\def\maxheight{\ifdim\Gin@nat@height>\textheight\textheight\else\Gin@nat@height\fi}
\makeatother
% Scale images if necessary, so that they will not overflow the page
% margins by default, and it is still possible to overwrite the defaults
% using explicit options in \includegraphics[width, height, ...]{}
\setkeys{Gin}{width=\maxwidth,height=\maxheight,keepaspectratio}
% Set default figure placement to htbp
\makeatletter
\def\fps@figure{htbp}
\makeatother
\setlength{\emergencystretch}{3em} % prevent overfull lines
\providecommand{\tightlist}{%
  \setlength{\itemsep}{0pt}\setlength{\parskip}{0pt}}
\setcounter{secnumdepth}{5}
\chapterstyle{bianchi}

\usepackage{mathtools}
\usepackage[italian]{babel} 
\usepackage{booktabs}
\usepackage{hyperref}
\hypersetup{
  colorlinks=true
}
\usepackage[
  labelfont=bf, 
  font={small, it} 
]{caption} 
\usepackage{upquote} % print correct quotes in verbatim-environments
\usepackage[autostyle, italian=quotes]{csquotes}
\usepackage{empheq} 
\usepackage{xfrac}

\raggedbottom % allow variable (ragged) site heights
\frenchspacing
% \setlength\parskip{1.5pt plus 1pt minus 0.5pt}


\DeclareMathOperator{\Var}{Var} % Define variance operator
\DeclareMathOperator{\SD}{SD} % Define sd operator
\DeclareMathOperator{\Cov}{Cov} % Define covariance operator
\DeclareMathOperator{\Corr}{Corr} % Define correlation operator
\DeclareMathOperator{\Me}{Me} % Define mediane operator
\DeclareMathOperator{\Mo}{Mo} % Define mode operator
\DeclareMathOperator{\Bin}{Bin} % Define binomial operator
\DeclareMathOperator{\Bernoulli}{Bernoulli} % Define Bernoulli operator
\DeclareMathOperator{\Poi}{Poi} % Define Poisson operator
\DeclareMathOperator{\Uniform}{Uniform} % Define Uniform operator
\DeclareMathOperator{\Cauchy}{Cauchy} % Define Cauchy operator
\DeclareMathOperator{\elpd}{elpd} % Define elpd operator
\DeclareMathOperator{\lppd}{lppd} % Define lppd operator
\DeclareMathOperator{\LOO}{LOO} % Define LOO operator
\DeclareMathOperator{\B}{\mathscr{B}} % Define Bernoulli operator
\newcommand{\R}{\textsf{R}} % Define R programming language symbol
\newcommand{\E}{\mathbb{E}} % Define expected value operator
\newcommand{\Real}{\mathbb{R}} % Define real number operator
\newcommand{\Prob}{\mathscr{P}}
\DeclareMathOperator*{\argmin}{arg\,min} % thin space, limits on side in displays
\DeclareMathOperator*{\argmax}{arg\,max} % thin space, limits on side in displays

\usepackage{microtype}

\ifLuaTeX
  \usepackage{selnolig}  % disable illegal ligatures
\fi
\usepackage[]{natbib}
\bibliographystyle{apalike}

\usepackage{amsthm}
\newtheorem{theorem}{Teorema}[chapter]
\newtheorem{lemma}{Lemma}[chapter]
\newtheorem{corollary}{Corollario}[chapter]
\newtheorem{proposition}{Proposizione}[chapter]
\newtheorem{conjecture}{Congettura}[chapter]
\theoremstyle{definition}
\newtheorem{definition}{Definizione}[chapter]
\theoremstyle{definition}
\newtheorem{example}{Esempio}[chapter]
\theoremstyle{definition}
\newtheorem{exercise}{Exercizio}[chapter]
\theoremstyle{definition}
\newtheorem{hypothesis}{Hypothesis}[chapter]
\theoremstyle{remark}
\newtheorem*{remark}{Osservazione}
\newtheorem*{solution}{Soluzione}
\begin{document}
\maketitle

{
\hypersetup{linkcolor=}
\setcounter{tocdepth}{1}
\tableofcontents
}
\newpage

\vspace*{5cm}

\thispagestyle{empty}

\hypertarget{sommatorie}{%
\chapter{Simbolo di somma (sommatorie)}\label{sommatorie}}

Le somme si incontrano costantemente in svariati contesti matematici e statistici quindi abbiamo bisogno di una notazione adeguata che ci consenta di gestirle. La somma dei primi \(n\) numeri interi può essere scritta come \(1+2+\dots+(n-1)+n\), dove `\(\dots\)' ci dice di completare la sequenza definita dai termini che vengono prima e dopo. Ovviamente, una notazione come \(1+7+\dots+73.6\) non avrebbe alcun senso senza qualche altro tipo di precisazione. In generale, nel seguito incontreremo delle somme nella forma
\begin{equation}
x_1+x_2+\dots+x_n,\notag
\end{equation}
dove \(x_i\) è un numero che è stato definito altrove. La notazione precedente, che fa uso dei tre puntini di sospensione, è utile in alcuni contesti ma in altri risulta ambigua. Pertanto la notazione di uso corrente è del tipo
\begin{equation}
  \sum_{i=1}^n x_i\notag
\end{equation}
\noindent
e si legge ``sommatoria per \(i\) che va da \(1\) a \(n\) di \(x_i\).'\,' Il simbolo \(\sum\) (lettera sigma maiuscola dell'alfabeto greco) indica l'operazione di somma, il simbolo \(x_i\) indica il generico addendo della sommatoria, le lettere \(1\) ed \(n\) indicano i cosiddetti \emph{estremi della sommatoria}, ovvero l'intervallo (da \(1\) fino a \(n\) estremi inclusi) in cui deve variare l'indice \(i\) allorché si sommano gli addendi \(x_i\).
Solitamente l'estremo inferiore è \(1\) ma potrebbe essere qualsiasi altri numero \(m < n\). Quindi
\[
  \sum_{i=1}^n x_i = x_1 + x_{2} + \dots + x_{n}.
\]
Per esempio, se i valori \(x\) sono \(\{3, 11, 4, 7\}\), si avrà
\[
  \sum_{i=1}^4 x_i = 3+11+4+7 = 25 
\]
laddove \(x_1 = 3\), \(x_2 = 11\), eccetera. La quantità \(x_i\) nella formula precedente si dice l'\emph{argomento} della sommatoria, mentre la variabile \(i\), che prende i valori naturali successivi indicati nel simbolo, si dice \emph{indice} della sommatoria.

La notazione di sommatoria può anche essere fornita nella forma seguente
\begin{equation}
  \sum_{P(i)} x_i\notag
\end{equation}
dove \(P(i)\) è qualsiasi proposizione riguardante \(i\) che può essere vera o falsa. Quando è ovvio che si vogliono sommare tutti i valori di \(n\) osservazioni, la notazione può essere semplificata nel modo seguente: \(\sum_{i} x_i\) oppure \(\sum x_i\). Al posto di \(i\) si possono trovare altre lettere: \(k, j, l, \dots\),.

\hypertarget{manipolazione-di-somme}{%
\section{Manipolazione di somme}\label{manipolazione-di-somme}}

È conveniente utilizzare le seguenti regole per semplificare i calcoli che coinvolgono l'operatore della sommatoria.

\hypertarget{proprietuxe0-1}{%
\subsection{Proprietà 1}\label{proprietuxe0-1}}

La sommatoria di \(n\) valori tutti pari alla stessa costante \(a\) è pari a \(n\) volte la costante stessa:
\[
  \sum_{i=1}^{n} a =  \underbrace{a + a + \dots + a}_{n~\text{volte}} = n a.
  \]

\hypertarget{proprietuxe0-2-proprietuxe0-distributiva}{%
\subsection{Proprietà 2 (proprietà distributiva)}\label{proprietuxe0-2-proprietuxe0-distributiva}}

Nel caso in cui l'argomento contenga una costante, è possibile riscrivere la sommatoria. Ad esempio con
\[
  \sum_{i=1}^{n} a x_i =  a x_1 + a x_2 + \dots + a x_n
  \]
è possibile raccogliere la costante \(a\) e fare \(a(x_1 +x_2 + \dots + x_n)\). Quindi possiamo scrivere
\[
  \sum_{i=1}^{n} a x_i =  a  \sum_{i=1}^{n} x_i.
  \]

\hypertarget{proprietuxe0-3-proprietuxe0-associativa}{%
\subsection{Proprietà 3 (proprietà associativa)}\label{proprietuxe0-3-proprietuxe0-associativa}}

Nel caso in cui
\[
  \sum_{i=1}^{n} (a + x_i) =  (a + x_1) +  (a + x_1) + \dots  (a + x_n)
  \]
si ha che
\[
  \sum_{i=1}^{n} (a + x_i) =  n a + \sum_{i=1}^{n} x_i.
  \]
È dunque chiaro che in generale possiamo scrivere
\[
  \sum_{i=1}^{n} (x_i + y_i) =  \sum_{i=1}^{n} x_i + \sum_{i=1}^{n} y_i.
  \]

\hypertarget{proprietuxe0-4}{%
\subsection{Proprietà 4}\label{proprietuxe0-4}}

Se deve essere eseguita un'operazione algebrica (innalzamento a potenza, logaritmo, ecc.) sull'argomento della sommatoria, allora tale operazione algebrica deve essere eseguita prima della somma. Per esempio,
\[
\sum_{i=1}^{n} x_i^2 = x_1^2 + x_2^2 + \dots + x_n^2 \neq \left(\sum_{i=1}^{n} x_i \right)^2.
\]

\hypertarget{proprietuxe0-5}{%
\subsection{Proprietà 5}\label{proprietuxe0-5}}

Nel caso si voglia calcolare \(\sum_{i=1}^{n} x_i y_i\), il prodotto tra i punteggi appaiati deve essere eseguito prima e la somma dopo:
\[
\sum_{i=1}^{n} x_i y_i = x_1 y_1 + x_2 y_2 + \dots + x_n y_n,
\]
infatti, \(a_1 b_1 + a_2 b_2 \neq (a_1 + a_2)(b_1 + b_2)\).

\hypertarget{doppia-sommatoria}{%
\section{Doppia sommatoria}\label{doppia-sommatoria}}

È possibile incontrare la seguente espressione in cui figurano una doppia sommatoria e un doppio indice:
\[
\sum_{i=1}^{n}\sum_{j=1}^{m} x_{ij}.
\]
La doppia sommatoria comporta che per ogni valore dell'indice esterno, \(i\) da \(1\) ad \(n\), occorre sviluppare la seconda sommatoria per \(j\) da \(1\) ad \(m\). Quindi,
\[
\sum_{i=1}^{3}\sum_{j=4}^{6} x_{ij} = (x_{1, 4} + x_{1, 5} + x_{1, 6}) + (x_{2, 4} + x_{2, 5} + x_{2, 6}) + (x_{3, 4} + x_{3, 5} + x_{3, 6}).
\]

Un caso particolare interessante di doppia sommatoria è il seguente:
\[
\sum_{i=1}^{n}\sum_{j=1}^{n} x_i y_j
\]
Si può osservare che nella sommatoria interna (quella che dipende dall'indice \(j\)), la quantità \(x_i\) è costante, ovvero non dipende dall'indice (che è \(j\)). Allora possiamo estrarre \(x_i\) dall'operatore di sommatoria interna e scrivere
\[
\sum_{i=1}^{n} \left( x_i \sum_{j=1}^{n} y_j \right).
\]
Allo stesso modo si può osservare che nell'argomento della sommatoria esterna la quantità costituita dalla sommatoria in \(j\) non dipende dall'indice \(i\) e quindi questa quantità può essere estratta dalla sommatoria esterna. Si ottiene quindi
\[
\sum_{i=1}^{n}\sum_{j=1}^{n} x_i y_j = \sum_{i=1}^{n} \left( x_i \sum_{j=1}^{n} y_j \right) = \sum_{i=1}^{n}\ x_i \sum_{j=1}^{n} y_j.
\]

\begin{example}
Si verifichi quanto detto sopra nel caso particolare di \(x = \{2, 3, 1\}\) e \(y = \{1, 4, 9\}\), svolgendo prima la doppia sommatoria per poi verificare che quanto così ottenuto sia uguale al prodotto delle due sommatorie.

\begin{align}
\sum_{i=1}^3 \sum_{j=1}^3 x_i y_j &= x_1y_1 + x_1y_2 + x_1y_3 + 
x_2y_1 + x_2y_2 + x_2y_3 + 
x_3y_1 + x_3y_2 + x_3y_3 \notag\\
&= 2 \times (1+4+9) + 3 \times (1+4+9) + 2 \times (1+4+9) = 84,\notag
\end{align}
ovvero
\[
(2 + 3 + 1) \times (1+4+9) = 84.
\]
\end{example}

\hypertarget{sommatorie-e-produttorie-e-operazioni-vettoriali-in-r}{%
\section{\texorpdfstring{Sommatorie (e produttorie) e operazioni vettoriali in \texttt{R}}{Sommatorie (e produttorie) e operazioni vettoriali in R}}\label{sommatorie-e-produttorie-e-operazioni-vettoriali-in-r}}

Si noti che la notazione
\[
\sum_{n=0}^4 3n
\]
non è altro che un ciclo \texttt{for}

\begin{Shaded}
\begin{Highlighting}[]
\NormalTok{sum }\OtherTok{=} \DecValTok{0}\NormalTok{;}
\ControlFlowTok{for}\NormalTok{ (}\AttributeTok{n =} \DecValTok{0}\NormalTok{; n }\SpecialCharTok{\textless{}=} \DecValTok{4}\NormalTok{; n}\SpecialCharTok{++}\NormalTok{) \{}
\NormalTok{  sum }\SpecialCharTok{+}\ErrorTok{=} \DecValTok{3} \SpecialCharTok{*}\NormalTok{ n;}
\NormalTok{\}}
\end{Highlighting}
\end{Shaded}

scritto in C, oppure

\begin{Shaded}
\begin{Highlighting}[]
\NormalTok{sum }\OtherTok{\textless{}{-}} \DecValTok{0}
\ControlFlowTok{for}\NormalTok{ (n }\ControlFlowTok{in} \DecValTok{0}\SpecialCharTok{:}\DecValTok{4}\NormalTok{) \{}
\NormalTok{  sum }\OtherTok{=}\NormalTok{ sum }\SpecialCharTok{+} \DecValTok{3} \SpecialCharTok{*}\NormalTok{ n}
\NormalTok{\}}
\NormalTok{sum}
\CommentTok{\#\textgreater{} [1] 30}
\end{Highlighting}
\end{Shaded}

scritto in \texttt{R}. In maniera equivalente, e più semplice, possiamo scrivere

\begin{Shaded}
\begin{Highlighting}[]
\FunctionTok{sum}\NormalTok{(}\DecValTok{3} \SpecialCharTok{*}\NormalTok{ (}\DecValTok{0}\SpecialCharTok{:}\DecValTok{4}\NormalTok{))}
\CommentTok{\#\textgreater{} [1] 30}
\end{Highlighting}
\end{Shaded}

Allo stesso modo, la notazione
\[
\prod_{n=1}^{4} 2n
\]
è equivalente al ciclo \texttt{for}

\begin{Shaded}
\begin{Highlighting}[]
\NormalTok{prod }\OtherTok{\textless{}{-}} \DecValTok{1}
\ControlFlowTok{for}\NormalTok{ (n }\ControlFlowTok{in} \DecValTok{1}\SpecialCharTok{:}\DecValTok{4}\NormalTok{) \{}
\NormalTok{  prod }\OtherTok{\textless{}{-}}\NormalTok{ prod }\SpecialCharTok{*} \DecValTok{2} \SpecialCharTok{*}\NormalTok{ n}
\NormalTok{\}}
\NormalTok{prod}
\CommentTok{\#\textgreater{} [1] 384}
\end{Highlighting}
\end{Shaded}

il che si può scrivere, più semplicemente, come

\begin{Shaded}
\begin{Highlighting}[]
\FunctionTok{prod}\NormalTok{(}\DecValTok{2} \SpecialCharTok{*}\NormalTok{ (}\DecValTok{1}\SpecialCharTok{:}\DecValTok{4}\NormalTok{))}
\CommentTok{\#\textgreater{} [1] 384}
\end{Highlighting}
\end{Shaded}


  \bibliography{refs.bib,book.bib,packages.bib}

\end{document}
