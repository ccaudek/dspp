% Template adapted from https://github.com/jgm/pandoc-templates/blob/master/default.latex
% To be used with XeLaTex in memoiR
%%%%%%%%%%%%%%%%%%%%%%%%%%%%%%%%%%%%%%%%%%%%%%%%%%%%%%%%%%%%%%%%%%%%%%%%%%%%%%%%%%%%%%%%%

% Options for packages loaded elsewhere
\PassOptionsToPackage{unicode=true}{hyperref}
\PassOptionsToPackage{hyphens}{url}
\PassOptionsToPackage{dvipsnames,svgnames*,x11names*}{xcolor}
% Right to left support


\documentclass[
  10pt,
  italian,
  a4paper,
  extrafontsizes,onecolumn,openright
  ]{memoir}

% Double (or whatever) spacing

% Math
\usepackage{amssymb, amsmath}
% mathspec: arbitrary math fonts
\usepackage{unicode-math}
\defaultfontfeatures{Scale=MatchLowercase}
\defaultfontfeatures[\rmfamily]{Ligatures=TeX,Scale=1}

% Fonts
\usepackage{lmodern}
\usepackage{fontspec}

% Main font
% Specific sanserif font
% Specific monotype font
\setmonofont[Scale=0.75]{Operator Mono SSm Lig Book}
% Specific math font
% Chinese, Japanese, Corean fonts

% Use upquote for straight quotes in verbatim environments
\usepackage{upquote}
% Use microtype
\usepackage[]{microtype}
\UseMicrotypeSet[protrusion]{basicmath} % disable protrusion for tt fonts

% Verbatim in note

% Color links
\usepackage{xcolor}

% Strikeout

% Necessary for code chunks

% Listings package

% Tables
\usepackage{longtable,booktabs,tabu}
% Fix footnotes in tables (requires footnote package)
\IfFileExists{footnote.sty}{\usepackage{footnote}\makesavenoteenv{longtable}}{}

% Graphics
\usepackage{graphicx,grffile}
\graphicspath{{images/}}
\makeatletter
\def\maxwidth{\ifdim\Gin@nat@width>\linewidth\linewidth\else\Gin@nat@width\fi}
\def\maxheight{\ifdim\Gin@nat@height>\textheight\textheight\else\Gin@nat@height\fi}
\makeatother
% Scale images if necessary, so that they will not overflow the page
% margins by default, and it is still possible to overwrite the defaults
% using explicit options in \includegraphics[width, height, ...]{}
\setkeys{Gin}{width=\maxwidth,height=\maxheight,keepaspectratio}

% Prevent overfull lines
\setlength{\emergencystretch}{3em}  
\providecommand{\tightlist}{%
  \setlength{\itemsep}{0pt}\setlength{\parskip}{0pt}}

% Number sections for memoir (secnumdepth counter is ignored)
\setsecnumdepth{section}

% Set default figure placement to htbp
\makeatletter
\def\fps@figure{htbp}
\makeatother

% Spacing in lists
\usepackage{enumitem}

% Polyglossia
\usepackage{polyglossia}
\setmainlanguage{it}
\setotherlanguage{en-US}

% BibLaTeX
\usepackage[backend=biber,style=authoryear-ibid,isbn=false,backref=true,giveninits=true,uniquename=init,maxcitenames=2,maxbibnames=150,sorting=nyt,sortcites=false,style=apa]{biblatex}
\addbibresource{refs.bib}

% cslreferences environment required by pandoc > 2.7



%%%%%%%%%%%%%%%%%%%%%%%%%%%%%%%%%%%%%%%%%%%%%%%%%%%%%%%%%%
% memoiR format

% Chapter Summary environment 
\usepackage[tikz]{bclogo}
\newenvironment{Summary}
  {\begin{bclogo}[logo=\bctrombone, noborder=true, couleur=lightgray!50]{In breve}\parindent0pt}
  {\end{bclogo}}
% Syntax:
%
%```{block, type='Summary'}
% Deliver message here.
% ```

% scriptsize code 
\let\oldverbatim\verbatim
\def\verbatim{\oldverbatim\scriptsize}
% Applies to code blocks and R code results
% code chunk options size='scriptsize' applies only to R code and results
% if the code chunk sets a different size, \def\verbatim{...} is prioritary for code results 


% Layout
%%%%%%%%%%%%%%%%%%%%%%%%%%%%%%%%%%%%%%%%%%%%%%%%%%%%%%%%%%

% Based on memoir, style companion
\newcommand{\MemoirChapStyle}{daleif1}
\newcommand{\MemoirPageStyle}{Ruled}

% Space between paragraphs
\usepackage{parskip}
  \abnormalparskip{3pt}

% Adjust margin paragraphs vertical position
\usepackage{marginfix}


% Margins
%%%%%%%%%%%%%%%%%%%%%%%%%%%%%%%%%%%%%%%

% allow use of '-',+','/' ans '*' to make simple length computation
\usepackage{calc}

% Full-width figures utilities
\newlength\widthw % full width
\newlength{\rf}
\newcommand*{\definesHSpace}{
  \strictpagecheck % slower but efficient detection of odd/even pages
  \checkoddpage
  \ifoddpage
  \setlength{\rf}{0mm}
  \else
  \setlength{\rf}{\marginparsep+\marginparwidth}
  \fi
}

\makeatletter
% 1" margins for the front matter.
\newcommand*{\SmallMargins}{
  \setlrmarginsandblock{1.5in}{1.5in}{*}
  \setmarginnotes{0.1in}{0.1in}{0.1in}
 \setulmarginsandblock{1.5in}{1in}{*}
  \checkandfixthelayout
  \ch@ngetext
  \clearpage
  \setlength{\widthw}{\textwidth+\marginparsep+\marginparwidth}
  \footnotesatfoot
  \chapterstyle{\MemoirChapStyle}  % Chapter and page styles must be recalled
  \pagestyle{\MemoirPageStyle}
}

% 3" outer margin for the main matter
\newcommand{\LargeMargins}{\SmallMargins}
\makeatother

% Figure captions and footnotes in outer margins


% Main title page with filigrane
%%%%%%%%%%%%%%%%%%%%%%%%%%%%%%%%%%%%%%%%%%%%%%%%%%%%%%%%%%

% Text blocks
\usepackage[absolute,overlay]{textpos}
  \setlength{\TPHorizModule}{1mm}
  \setlength{\TPVertModule}{1mm}

\newcommand{\MainTitlePage}[2]{
  \SmallMargins % Margins
  \pagestyle{empty} % No header/footer
  \textblockorigin{\stockwidth-\paperwidth-\trimedge}{\trimtop} % recto
  \begin{textblock*}{2mm}(\spinemargin/2,\uppermargin/2)
    \rule{1pt}{\paperheight-\uppermargin}
  \end{textblock*}
  \begin{textblock*}{\paperwidth*2/3}(\paperwidth/5, \paperheight/5)
    \flushright
    \begin{Spacing}{3}
      {\fontfamily{qtm}\selectfont\fontsize{45}{45}\selectfont\textsc{\thetitle}}
    \end{Spacing}
  \end{textblock*}
    \begin{textblock*}{\paperwidth*2/3}(\paperwidth/5, \paperheight/2)
    \flushright
    {\fontfamily{qtm}\huge\theauthor}
  \end{textblock*}
    \begin{textblock*}{\paperwidth*2/3}[0, 1](\spinemargin, \uppermargin+\textheight)
    \normalfont\thedate
  \end{textblock*}
  ~\\ % Print a character or the page will not exist
  \newpage
  \textblockorigin{\trimedge}{\trimtop} % verso
  \begin{textblock*}{\textwidth}(\paperwidth-\spinemargin-\textwidth, \uppermargin)
    #1
  \end{textblock*}
  \begin{textblock*}{\textwidth}[0,1](\paperwidth-\spinemargin-\textwidth, \uppermargin+\textheight+\footskip)
    \centering
    \includegraphics[width=\paperwidth/4]{logo}\\ \bigskip
    #2
  \end{textblock*}
  ~\\ % Print a character or the page will not exist
  \newpage
}

% Clear page and open an even one (\clearpage opens an odd one)
\newcommand{\evenpage}{
  \clearpage
  \strictpagecheck % slower but efficient detection of odd/even pages
  \checkoddpage
  \ifoddpage
    \thispagestyle{empty}
    ~\\ % Print a character or the page will not exist
    \newpage
  \else
    % do nothing
  \fi
}


%% PDF title page to insert
%%%%%%%%%%%%%%%%%%%%%%%%%%%%%%%%%%%%%%%%%%%%%%%%%%%%%%%%%%



%% Bibliography
%%%%%%%%%%%%%%%%%%%%%%%%%%%%%%%%%%%%%%%%%%%%%%%%%%%%%%%%%%

\usepackage[strict,autostyle]{csquotes}
% Repeated citation as author-year-title instead of author-title (modification of footcite:note in verbose-inote.cbx)

%% Table of Contents
%%%%%%%%%%%%%%%%%%%%%%%%%%%%%%%%%%%%%%%%%%%%%%%%%%%%%%%%%%

% fix the typesetting of the part number
\renewcommand\partnumberlinebox[2]{#2\ ---\ }


% Fonts
%%%%%%%%%%%%%%%%%%%%%%%%%%%%%%%%%%%%%%%%%%%%%%%%%%%%%%%%%%


% Hyperref comes last
%%%%%%%%%%%%%%%%%%%%%%%%%%%%%%%%%%%%%%%%%%%%%%%%%%%%%%%%%%

\usepackage{hyperref}
\hypersetup{
  pdftitle={Psicometria},
  pdfauthor={Corrado Caudek},
  colorlinks=true,
  linkcolor=Maroon,
  citecolor=Blue,
  urlcolor=Blue,
  breaklinks=true}

% Don't use monospace font for urls
\urlstyle{same}


% Title, author, date from YAML to LaTeX
%%%%%%%%%%%%%%%%%%%%%%%%%%%%%%%%%%%%%%%%%%%%%%%%%%%%%%%%%%

\title{Psicometria}

\author{Corrado Caudek}

\date{2021-12-18}


% Include headers (preamble.tex) here
%%%%%%%%%%%%%%%%%%%%%%%%%%%%%%%%%%%%%%%%%%%%%%%%%%%%%%%%%%
% Add LaTeX code into the preamble of the document here
\hyphenation{bio-di-ver-si-ty sap-lings}


%%%%%%%%%%%%%%%%%%%%%%%%%%%%%%%%%%%%%%%%%%%%%%%%%%%%%%%%%%%%%%%%%%%%%%%%%
% memoiR dalef3 chapter style 
% https://ctan.crest.fr/tex-archive/info/latex-samples/MemoirChapStyles/MemoirChapStyles.pdf
\usepackage{soul}
\definecolor{nicered}{rgb}{.647,.129,.149}

\makeatletter
\makechapterstyle{pedersen}
\makeatother



%\makeatletter
%\newlength\dlf@normtxtw
%\setlength\dlf@normtxtw{\textwidth}
%\def\myhelvetfont{\def\sfdefault{mdput}}
%\newsavebox{\feline@chapter}
%\newcommand\feline@chapter@marker[1][4cm]{%
%  \sbox\feline@chapter{%
%    \resizebox{!}{#1}{\fboxsep=1pt%
%	  \colorbox{nicered}{\color{white}\bfseries\sffamily\thechapter}%
%	}}%
%  \rotatebox{90}{%
%    \resizebox{%
%	  \heightof{\usebox{\feline@chapter}}+\depthof{\usebox{\feline@chapter}}}%
%	{!}{\scshape\so\@chapapp}}\quad%
%  \raisebox{\depthof{\usebox{\feline@chapter}}}{\usebox{\feline@chapter}}%
% }
%\newcommand\feline@chm[1][4cm]{%
%  \sbox\feline@chapter{\feline@chapter@marker[#1]}%
%  \makebox[0pt][l]{% aka \rlap
%    \makebox[1cm][r]{\usebox\feline@chapter}%
%  }}
%\makechapterstyle{pedersen}{ %daleif1
%
%  \renewcommand\chapnamefont{\normalfont\Large\scshape\raggedleft\so}
%  
%  % I changed this!!
%  %\renewcommand\chaptitlefont{\normalfont\huge\bfseries\scshape\color{nicered}}
%  \renewcommand\chaptitlefont{\normalfont\huge\fontencoding{T1}\fontfamily{phv}\selectfont\color{nicered}}
%    
%  \renewcommand\chapternamenum{}
%  \renewcommand\printchaptername{}
%  \renewcommand\printchapternum{\null\hfill\feline@chm[2.5cm]\par}
%  \renewcommand\afterchapternum{\par\vskip\midchapskip}
%  \renewcommand\printchaptertitle[1]{\chaptitlefont\raggedleft ##1\par}
%}
%\makeatother

\DeclareMathOperator{\Var}{Var} % Define variance operator
\DeclareMathOperator{\SD}{SD} % Define sd operator
\DeclareMathOperator{\Cov}{Cov} % Define covariance operator
\DeclareMathOperator{\Corr}{Corr} % Define correlation operator
\DeclareMathOperator{\Me}{Me} % Define mediane operator
\DeclareMathOperator{\Mo}{Mo} % Define mode operator
\DeclareMathOperator{\Bin}{Bin} % Define binomial operator
\DeclareMathOperator{\Bernoulli}{Bernoulli} % Define Bernoulli operator
\DeclareMathOperator{\Poi}{Poi} % Define Poisson operator
\DeclareMathOperator{\Uniform}{Uniform} % Define Uniform operator
\DeclareMathOperator{\Cauchy}{Cauchy} % Define Cauchy operator
\DeclareMathOperator{\elpd}{elpd} % Define elpd operator
\DeclareMathOperator{\lppd}{lppd} % Define lppd operator
\DeclareMathOperator{\LOO}{LOO} % Define LOO operator
\DeclareMathOperator{\B}{\mathscr{B}} % Define Bernoulli operator
\newcommand{\R}{\textsf{R}} % Define R programming language symbol
\newcommand{\E}{\mathbb{E}} % Define expected value operator
\newcommand{\Real}{\mathbb{R}} % Define real number operator
\newcommand{\Prob}{\mathscr{P}}
\DeclareMathOperator*{\argmin}{arg\,min} % thin space, limits on side in displays
\DeclareMathOperator*{\argmax}{arg\,max} % thin space, limits on side in displays

\raggedbottom % allow variable (ragged) site heights
\frenchspacing

\usepackage[
  labelfont=bf, 
  font={small, it} 
]{caption} 
\usepackage{upquote} % print correct quotes in verbatim-environments
\usepackage{empheq} 
\usepackage{xfrac}
\usepackage{lstbayes}


% Introduction to Modern statistics ------------------------------------------------
% https://github.com/OpenIntroStat/ims/blob/main/latex/ims-style.tex

\usepackage[framemethod=tikz]{mdframed} 
\usepackage{helvet} 
\usepackage{xcolor}


\definecolor{oiB}{HTML}{000000}            % COL["blue","full"]
\definecolor{oiLB}{HTML}{e0e0e0}           % lighter version of oiB

\definecolor{oiY}{HTML}{000000}            % COL["yellow","full"]
\definecolor{oiLY}{HTML}{e0e0e0}           % lighter version of oiY

\definecolor{oiR}{HTML}{000000}            % COL["red","full"]
\definecolor{oiLR}{HTML}{e0e0e0}           % lighter version of oiR

\definecolor{oiGray}{HTML}{808080}         % COL["gray","full"]
\definecolor{oiLGray}{HTML}{f8f8f8}        % lighter version of oiR

\definecolor{oiGB}{rgb}{0.5,0.5,.5}        % from OS4 - for footnotes


% Helper environments ------------------------------------------------------------

% mdframedwithfootChapterintro: for chapterintro box

\newenvironment{mdframedwithfootChapterintro}
{   
    \savenotes
    \begin{mdframed}[%
    topline=true, bottomline=true, linecolor=oiB, linewidth=1.4pt,
    rightline=false, leftline=false,
    backgroundcolor=oiLB]
    %\stepcounter{footnote} % don't increment footnote counter
    \renewcommand{\thempfootnote}{\arabic{footnote}}
    }
{
    \end{mdframed}
    \spewnotes
}


% mdframedwithfootGPWE: for guidedpractice and workedexample

\newenvironment{mdframedwithfootGPWE}
{   
    \savenotes
    \begin{mdframed}[%
    topline=true, bottomline=true, linecolor=oiB, linewidth=0.5pt,
    rightline=false, leftline=false,
    backgroundcolor=oiLGray]
    %\stepcounter{footnote}
    \renewcommand{\thempfootnote}{\arabic{footnote}}
    }
{
    \end{mdframed}
    \spewnotes
}


% mdframedwithfootImportant: for important

\newenvironment{mdframedwithfootImportant}
{   
    \savenotes
    \begin{mdframed}[%
    topline=true, bottomline=true, linecolor=oiR, linewidth=0.5pt,
    rightline=false, leftline=false,
    backgroundcolor=oiLGray]
    %\stepcounter{footnote}
    \renewcommand{\thempfootnote}{\arabic{footnote}}
    }
{
    \end{mdframed}
    \spewnotes
}


% mdframedwithfootTip: for tip, data, and pronunciation

\newenvironment{mdframedwithfootTipDataPro}
{   
    \savenotes
    \begin{mdframed}[%
    topline=true, bottomline=true, linecolor=oiGray, linewidth=0.5pt,
    rightline=false, leftline=false,
    backgroundcolor=oiLGray]
    %\stepcounter{footnote}
    \renewcommand{\thempfootnote}{\arabic{footnote}}
    }
{
    \end{mdframed}
    \spewnotes
}


% Custom environments/boxes -------------------------------------------------------

% chapterintro

\newenvironment{chapterintro}{
\vspace{4mm}
\begin{mdframedwithfootChapterintro}
\begin{minipage}[t]{0.10\textwidth}
{$\:$ \\ \setkeys{Gin}{width=2.5em,keepaspectratio}\includegraphics{images/_icons/chapterintro.png}}
\end{minipage}
\hfill
\begin{minipage}[t]{0.90\textwidth}
\setlength{\parskip}{1em}
\large
}{\end{minipage}
\end{mdframedwithfootChapterintro}
\vspace{4mm}
}

% guidedpractice

\newenvironment{guidedpractice}{
\vspace{4mm}
\begin{mdframedwithfootGPWE}
\begin{minipage}[t]{0.10\textwidth}
{$\:$ \\ \setkeys{Gin}{width=2.5em,keepaspectratio}\includegraphics{images/_icons/guided-practice.png}}
\end{minipage}
\hfill
\begin{minipage}[t]{0.90\textwidth}
\vspace{-2mm}
\setlength{\parskip}{1em}
\noindent\textbf{\color{oiB}\small\fontencoding{T1}\fontfamily{phv}\selectfont{\MakeUppercase{Pratica guidata}}} $\:$ \\ \\
}{\end{minipage}
\end{mdframedwithfootGPWE}
\vspace{4mm}
}


% workedexample

\newenvironment{workedexample}{
    \let\oldrule\rule
    \renewcommand{\rule}[2]{\vspace{-2mm}\oldrule{##1}{##2}\vspace{-2mm}}
\vspace{4mm}
\begin{mdframedwithfootGPWE}
\begin{minipage}[t]{0.10\textwidth}
{$\:$ \\ \setkeys{Gin}{width=2.5em,keepaspectratio}\includegraphics{images/_icons/worked-example.png}}
\end{minipage}
\hfill
\begin{minipage}[t]{0.90\textwidth}
\vspace{-2mm}
\setlength{\parskip}{1em}
\noindent\textbf{\color{oiB}\small\fontencoding{T1}\fontfamily{phv}\selectfont{\MakeUppercase{Esempio}}} $\:$ \\ \\
}{\end{minipage}
\end{mdframedwithfootGPWE}
\vspace{4mm}
}


% important

\newenvironment{important}{
    \let\oldtextbf\textbf
    \renewcommand{\textbf}[1]{{\textcolor{oiR}{\oldtextbf{##1}}}}
\vspace{4mm}
\begin{mdframedwithfootImportant}
\begin{minipage}[t]{0.10\textwidth}
{$\:$ \\ \setkeys{Gin}{width=2.5em,keepaspectratio}\includegraphics{images/_icons/important.png}}
\end{minipage}
\hfill
\begin{minipage}[t]{0.90\textwidth}
\vspace{-2mm}
\setlength{\parskip}{1em}
}{\end{minipage}
\end{mdframedwithfootImportant}
\vspace{4mm}
}

% tip

\newenvironment{tip}{
\vspace{4mm}
\begin{mdframedwithfootTipDataPro}
\begin{minipage}[t]{0.10\textwidth}
{$\:$ \\ \setkeys{Gin}{width=2em,keepaspectratio}\includegraphics{images/_icons/tip.png}}
\end{minipage}
\hfill
\begin{minipage}[t]{0.90\textwidth}
\vspace{-2mm}
\setlength{\parskip}{1em}
}{\end{minipage}
\end{mdframedwithfootTipDataPro}
\vspace{4mm}
}

% data

\newenvironment{data}{
\vspace{4mm}
\begin{mdframedwithfootTipDataPro}
\begin{minipage}[t]{0.10\textwidth}
{$\:$ \\ \setkeys{Gin}{width=2em,keepaspectratio}\includegraphics{images/_icons/data.png}}
\end{minipage}
\hfill
\begin{minipage}[t]{0.90\textwidth}
\vspace{-2mm}
\setlength{\parskip}{1em}
}{\end{minipage}
\end{mdframedwithfootTipDataPro}
\vspace{4mm}
}

%\usepackage{titlesec}
%\titleformat{\chapter}[display]
%  {\normalfont\sffamily\huge\bfseries\color{blue}}
%  {\chaptertitlename\ \thechapter}{20pt}{\Huge}
%\titleformat{\section}
%  {\normalfont\sffamily\Large\bfseries\color{cyan}}
%  {\thesection}{1em}{}

%%%%%%%%%%%%%%%%%%%%%%%%%%%%


\usepackage{booktabs}
\usepackage{longtable}
\usepackage{array}
\usepackage{multirow}
\usepackage{wrapfig}
\usepackage{float}
\usepackage{colortbl}
\usepackage{pdflscape}
\usepackage{tabu}
\usepackage{threeparttable}
\usepackage{threeparttablex}
\usepackage[normalem]{ulem}
\usepackage{makecell}
\usepackage{xcolor}


% End of preamble
%%%%%%%%%%%%%%%%%%%%%%%%%%%%%%%%%%%%%%%%%%%%%%%%%%%%%%%%%%


\begin{document}
\frontmatter

% Title page
%%%%%%%%%%%%%%%%%%%%%%%%%%%%%%%%%%%%%%%%%%%%%%%%%%%%%%%%%%


\MainTitlePage{Questo documento è stato realizzato con:

\begin{itemize}
  \item \LaTeX\; e la classe memoir (\url{http://www.ctan.org/pkg/memoir});
  \item $\R$ (\url{http://www.r-project.org/}) e RStudio (\url{http://www.rstudio.com/});
  \item bookdown (\url{http://bookdown.org/}) e memoiR (\url{https://ericmarcon.github.io/memoiR/}).
\end{itemize}}{Nel blog della mia pagina personale sono forniti alcuni approfondimenti degli argomenti qui trattati.

\url{https://ccaudek.github.io/caudeklab/}}


% Before Body
%%%%%%%%%%%%%%%%%%%%%%%%%%%%%%%%%%%%%%%%%%%%%%%%%%%%%%%%%%





% Contents
%%%%%%%%%%%%%%%%%%%%%%%%%%%%%%%%%%%%%%%%%%%%%%%%%%%%%%%%%%

\LargeMargins
{
\hypersetup{linkcolor=}
\setcounter{tocdepth}{2}
\tableofcontents
}


% Body
%%%%%%%%%%%%%%%%%%%%%%%%%%%%%%%%%%%%%%%%%%%%%%%%%%%%%%%%%%

\LargeMargins
\scriptsize

\normalsize

\chapter*{}

\vfill

\scriptsize

\normalsize

\scriptsize

Copyright \(\copyright\) 2022.

\normalsize

Data della versione presente: Dicembre 18, 2021.

\hypertarget{prefazione}{%
\chapter{Prefazione}\label{prefazione}}

\textbf{Data Science per psicologi} contiene il materiale delle lezioni dell'insegnamento di \emph{Psicometria B000286} (A.A. 2021/2022) rivolto agli studenti del primo anno del Corso di Laurea in Scienze e Tecniche Psicologiche dell'Università degli Studi di Firenze.

L'insegnamento di Psicometria si propone di fornire agli studenti un'introduzione all'analisi dei dati in psicologia.
Le conoscenze/competenze che verranno sviluppate in questo insegnamento sono quelle della \emph{Data science}, ovvero le conoscenze/competenze che si pongono all'intersezione tra statistica (ovvero, richiedono la capacità di comprendere teoremi statistici) e informatica (ovvero, richiedono la capacità di sapere utilizzare un software).

\hypertarget{la-psicologia-e-la-data-science}{%
\section*{La psicologia e la Data Science}\label{la-psicologia-e-la-data-science}}
\addcontentsline{toc}{section}{La psicologia e la Data Science}

\begin{quote}
It's worth noting, before getting started, that this material is hard. If you find yourself confused at any point, you are normal. Any sense of confusion you feel is just your brain correctly calibrating to the subject matter. Over time, confusion is replaced by comprehension {[}\ldots{]} --- Richard McElreath
\end{quote}

Sembra sensato spendere due parole su un tema che è importante per gli studenti: quello indicato dal titolo di questo Capitolo. È ovvio che agli studenti di psicologia la statistica non piace. Se piacesse, forse studierebbero Data Science e non psicologia; ma non lo fanno. Di conseguenza, gli studenti di psicologia si chiedono: ``perché dobbiamo perdere tanto tempo a studiare queste cose quando in realtà quello che ci interessa è tutt'altro?'\,' Questa è una bella domanda.

C'è una ragione molto semplice che dovrebbe farci capire perché la Data Science è così importante per la psicologia. Infatti, a ben pensarci, la psicologia è una disciplina intrinsecamente statistica, se per statistica intendiamo quella disciplina che studia la variazione delle caratteristiche degli individui nella popolazione. La psicologia studia \emph{gli individui} ed è proprio la variabilità inter- e intra-individuale ciò che vogliamo descrivere e, in certi casi, predire. In questo senso, la psicologia è molto diversa dall'ingegneria, per esempio. Le proprietà di un determinato ponte sotto certe condizioni, ad esempio, sono molto simili a quelle di un altro ponte, sotto le medesime condizioni. Quindi, per un ingegnere la statistica è poco importante: le proprietà dei materiali sono unicamente dipendenti dalla loro composizione e restano costanti. Ma lo stesso non può dirsi degli individui: ogni individuo è unico e cambia nel tempo. E le variazioni tra gli individui, e di un individuo nel tempo, sono l'oggetto di studio proprio della psicologia: è dunque chiaro che i problemi che la psicologia si pone sono molto diversi da quelli affrontati, per esempio, dagli ingegneri. Questa è la ragione per cui abbiamo tanto bisogno della \emph{data science} in psicologia: perché la \emph{data science} ci consente di descrivere la variazione e il cambiamento. E queste sono appunto le caratteristiche di base dei fenomeni psicologici.

Sono sicuro che, leggendo queste righe, a molti studenti sarà venuta in mente la seguente domanda: perché non chiediamo a qualche esperto di fare il ``lavoro sporco'' (ovvero le analisi statistiche) per noi, mentre noi (gli psicologi) ci occupiamo solo di ciò che ci interessa, ovvero dei problemi psicologici slegati dai dettagli ``tecnici'' della \emph{data science}?
La risposta a questa domanda è che non è possibile progettare uno studio psicologico sensato senza avere almeno una comprensione rudimentale della \emph{data science}. Le tematiche della \emph{data science} non possono essere ignorate né dai ricercatori in psicologia né da coloro che svolgono la professione di psicologo al di fuori dell'Università. Infatti, anche i professionisti al di fuori dall'università non possono fare a meno di leggere la letteratura psicologica più recente: il continuo aggiornamento delle conoscenze è infatti richiesto dalla deontologia della professione. Ma per potere fare questo è necessario conoscere un bel po' di \emph{data science}! Basta aprire a caso una rivista specialistica di psicologia per rendersi conto di quanto ciò sia vero: gli articoli che riportano i risultati delle ricerche psicologiche sono zeppi di analisi statistiche e di modelli formali. E la comprensione della letteratura psicologica rappresenta un requisito minimo nel bagaglio professionale dello psicologo.

Le considerazioni precedenti cercano di chiarire il seguente punto: la \emph{data science} non è qualcosa da studiare a malincuore, in un singolo insegnamento universitario, per poi poterla tranquillamente dimenticare. Nel bene e nel male, gli psicologi usano gli strumenti della \emph{data science} in tantissimi ambiti della loro attività professionale: in particolare quando costruiscono, somministrano e interpretano i test psicometrici. È dunque chiaro che possedere delle solide basi di \emph{data science} è un tassello imprescindibile del bagaglio professionale dello psicologo. In questo insegnamento verrano trattati i temi base della \emph{data science} e verrà adottato un punto di vista bayesiano, che corrisponde all'approccio più recente e sempre più diffuso in psicologia.

\hypertarget{come-studiare}{%
\section*{Come studiare}\label{come-studiare}}
\addcontentsline{toc}{section}{Come studiare}

\begin{quote}
I know quite certainly that I myself have no special talent. Curiosity, obsession and dogged endurance, combined with self-criticism, have brought me to my ideas. --- Albert Einstein
\end{quote}

Il giusto metodo di studio per prepararsi all'esame di Psicometria è quello di seguire attivamente le lezioni, assimilare i concetti via via che essi vengono presentati e verificare in autonomia le procedure presentate a lezione. Incoraggio gli studenti a farmi domande per chiarire ciò che non è stato capito appieno. Incoraggio gli studenti a utilizzare i forum attivi su Moodle e, soprattutto, a svolgere gli esercizi proposti su Moodle. I problemi forniti su Moodle rappresentano il livello di difficoltà richiesto per superare l'esame e consentono allo studente di comprendere se le competenze sviluppate fino a quel punto sono sufficienti rispetto alle richieste dell'esame.

La prima fase dello studio, che è sicuramente individuale, è quella in cui è necessario acquisire le conoscenze teoriche relative ai problemi che saranno presentati all'esame. La seconda fase di studio, che può essere facilitata da scambi con altri e da incontri di gruppo, porta ad acquisire la capacità di applicare le conoscenze: è necessario capire come usare un software (\R) per applicare i concetti statistici alla specifica situazione del problema che si vuole risolvere. Le due fasi non sono però separate: il saper fare molto spesso ci aiuta a capire meglio.

\hypertarget{sviluppare-un-metodo-di-studio-efficace}{%
\section*{Sviluppare un metodo di studio efficace}\label{sviluppare-un-metodo-di-studio-efficace}}
\addcontentsline{toc}{section}{Sviluppare un metodo di studio efficace}

\begin{quote}
Memorization is not learning. --- Richard Phillips Feynman
\end{quote}

Avendo insegnato molte volte in passato un corso introduttivo di analisi dei dati ho notato nel corso degli anni che gli studenti con l'atteggiamento mentale che descriverò qui sotto generalmente ottengono ottimi risultati. Alcuni studenti sviluppano naturalmente questo approccio allo studio, ma altri hanno bisogno di fare uno sforzo per maturarlo. Fornisco qui sotto una breve descrizione del ``metodo di studio'\,' che, nella mia esperienza, è il più efficace per affrontare le richieste di questo insegnamento \autocite{burger20125}.

\begin{itemize}
\tightlist
\item
  Dedicate un tempo sufficiente al materiale di base, apparentemente facile; assicuratevi di averlo capito bene. Cercate le lacune nella vostra comprensione. Leggere presentazioni diverse dello stesso materiale (in libri o articoli diversi) può fornire nuove intuizioni.
\end{itemize}

\begin{itemize}
\item
  Gli errori che facciamo sono i nostri migliori maestri. Istintivamente cerchiamo di dimenticare subito i nostri errori. Ma il miglior modo di imparare è apprendere dagli errori che commettiamo. In questo senso, una soluzione corretta è meno utile di una soluzione sbagliata. Quando commettiamo un errore questo ci fornisce un'informazione importante: ci fa capire qual è il materiale di studio sul quale dobbiamo ritornare e che dobbiamo capire meglio.
\item
  C'è ovviamente un aspetto ``psicologico'' nello studio. Quando un esercizio o problema ci sembra incomprensibile, la cosa migliore da fare è dire: ``mi arrendo'', ``non ho idea di cosa fare!''. Questo ci rilassa: ci siamo già arresi, quindi non abbiamo niente da perdere, non dobbiamo più preoccuparci. Ma non dobbiamo fermarci qui. Le cose ``migliori'' che faccio (se ci sono) le faccio quando non ho voglia di lavorare. Alle volte, quando c'è qualcosa che non so fare e non ho idea di come affontare, mi dico: ``oggi non ho proprio voglia di fare fatica'', non ho voglia di mettermi nello stato mentale per cui ``in 10 minuti devo risolvere il problema perché dopo devo fare altre cose''. Però ho voglia di \emph{divertirmi} con quel problema e allora mi dedico a qualche aspetto ``marginale'' del problema, che so come affrontare, oppure considero l'aspetto più difficile del problema, quello che non so come risolvere, ma invece di cercare di risolverlo, guardo come altre persone hanno affrontato problemi simili, opppure lo stesso problema in un altro contesto. Non mi pongo l'obiettivo ``risolvi il problema in 10 minuti'', ma invece quello di farmi un'idea ``generale'' del problema, o quello di capire un caso più specifico e più semplice del problema. Senza nessuna pressione. Infatti, in quel momento ho deciso di non lavorare (ovvero, di non fare fatica). Va benissimo se ``parto per la tangente'', ovvero se mi metto a leggere del materiale che sembra avere poco a che fare con il problema centrale (le nostre intuizioni e la nostra curiosità solitamente ci indirizzano sulla strada giusta). Quando faccio così, molto spesso trovo la soluzione del problema che mi ero posto e, paradossalmente, la trovo in un tempo minore di quello che, in precedenza, avevo dedicato a ``lavorare'' al problema. Allora perché non faccio sempre così? C'è ovviamente l'aspetto dei ``10 minuti'' che non è sempre facile da dimenticare. Sotto pressione, possiamo solo agire in maniera automatica, ovvero possiamo solo applicare qualcosa che già sappiamo fare. Ma se dobbiamo imparare qualcosa di nuovo, la pressione è un impedimento.
\item
  È utile farsi da soli delle domande sugli argomenti trattati, senza limitarsi a cercare di risolvere gli esercizi che vengono assegnati. Quando studio qualcosa mi viene in mente: ``se questo è vero, allora deve succedere quest'altra cosa''. Allora verifico se questo è vero, di solito con una simulazione. Se i risultati della simulazione sono quelli che mi aspetto, allora vuol dire che ho capito. Se i risultati sono diversi da quelli che mi aspettavo, allora mi rendo conto di non avere capito e ritorno indietro a studiare con più attenzione la teoria che pensavo di avere capito -- e ovviamente mi rendo conto che c'era un aspetto che avevo frainteso. Questo tipo di verifica è qualcosa che dobbiamo fare da soli, in prima persona: nessun altro può fare questo al posto nostro.
\item
  Non aspettatevi di capire tutto la prima volta che incontrate un argomento nuovo.\footnote{Ricordatevi inoltre che gli individui tendono a sottostimare la propria capacità di apprendere \autocite{horn2021underestimating}.} È utile farsi una nota mentalmente delle lacune nella vostra comprensione e tornare su di esse in seguito per carcare di colmarle. L'atteggiamento naturale, quando non capiamo i dettagli di qualcosa, è quello di pensare: ``non importa, ho capito in maniera approssimativa questo punto, non devo preoccuparmi del resto''. Ma in realtà non è vero: se la nostra comprensione è superficiale, quando il problema verrà presentato in una nuova forma, non riusciremo a risolverlo. Per cui i dubbi che ci vengono quando studiamo qualcosa sono il nostro alleato più prezioso: ci dicono esattamente quali sono gli aspetti che dobbiamo approfondire per potere migliorare la nostra preparazione.
\item
  È utile sviluppare una visione d'insieme degli argomenti trattati, capire l'obiettivo generale che si vuole raggiungere e avere chiaro il contributo che i vari pezzi di informazione forniscono al raggiungimento di tale obiettivo. Questa organizzazione mentale del materiale di studio facilita la comprensione. È estremamente utile creare degli schemi di ciò che si sta studiando. Non aspettate che sia io a fornirvi un riepilogo di ciò che dovete imparare: sviluppate da soli tali schemi e tali riassunti.
\item
  Tutti noi dobbiamo imparare l'arte di trovare le informazioni, non solo nel caso di questo insegnamento. Quando vi trovate di fronte a qualcosa che non capite, o ottenete un oscuro messaggio di errore da un software, ricordatevi: ``Google is your friend''.
\end{itemize}

\bigskip

Corrado Caudek

\bigskip

Febbraio 2022

\mainmatter

\hypertarget{part-nozioni-preliminari}{%
\part*{Nozioni preliminari}\label{part-nozioni-preliminari}}
\addcontentsline{toc}{part}{Nozioni preliminari}

\hypertarget{concetti-chiave}{%
\chapter{Concetti chiave}\label{concetti-chiave}}

\begin{chapterintro}
La \emph{data science} si pone all'intersezione tra statistica e informatica. La statistica è un insieme di metodi per estrarre informazioni dai dati; l'informatica implementa tali procedure in un software. In questo Capitolo vengono introdotti i concetti fondamentali.

\end{chapterintro}

\hypertarget{popolazioni-e-campioni}{%
\section{Popolazioni e campioni}\label{popolazioni-e-campioni}}

\textbf{Popolazione.} L'analisi dei dati inizia con l'individuazione delle unità portatrici di informazioni circa il fenomeno di interesse. Si dice popolazione (o universo) l'insieme \(\Omega\) delle entità capaci di fornire informazioni sul fenomeno oggetto dell'indagine statistica. Possiamo scrivere \(\Omega = \{\omega_i\}_{i=1, \dots, n}= \{\omega_1, \omega_2, \dots, \omega_n\}\), oppure \(\Omega = \{\omega_1, \omega_2, \dots \}\) nel caso di popolazioni finite o infinite, rispettivamente.

L'obiettivo principale della ricerca psicologica è conoscere gli esiti psicologici e i loro fattori trainanti nella popolazione. Questo è l'obiettivo delle sperimentazioni psicologiche e della maggior parte degli studi osservazionali in psicologia. È quindi necessario essere molto chiari sulla popolazione a cui si applicano i risultati della ricerca. La popolazione può essere ben definita, ad esempio, tutte le persone che si trovavano nella città di Hiroshima al momento dei bombardamenti atomici e sono sopravvissute al primo anno, o può essere ipotetica, ad esempio, tutte le persone depresse che hanno subito o saranno sottoporsi ad un intervento di psicoterapia. Il ricercatore deve sempre essere in grado di determinare se un soggetto appartiene alla popolazione oggetto di interesse.

Una \emph{sottopopolazione} è una popolazione in sé e per sé che soddisfa proprietà ben definite. Negli esempi precedenti, potremmo essere interessati alla sottopopolazione di uomini di età inferiore ai 20 anni o di pazienti depressi sottoposti ad uno specifico intervento psicologico. Molte questioni scientifiche riguardano le differenze tra sottopopolazioni; ad esempio, confrontando i gruppi con o senza psicoterapia per determinare se il trattamento è vantaggioso. I modelli di regressione, introdotti nel Capitolo \ref{regr-models-intro} riguardano le sottopopolazioni, in quanto stimano il risultato medio per diversi gruppi (sottopopolazioni) definiti dalle covariate.

\textbf{Campione.} Gli elementi \(\omega_i\) dell'insieme \(\Omega\) sono detti \emph{unità statistiche}. Un sottoinsieme della popolazione, ovvero un insieme di elementi \(\omega_i\), viene chiamato \emph{campione}. Ciascuna unità statistica \(\omega_i\) (abbreviata con u.s.) è portatrice dell'informazione che verrà rilevata mediante un'operazione di misurazione.

Un campione è dunque un sottoinsieme della popolazione utilizzato per conoscere tale popolazione. A differenza di una sottopopolazione definita in base a chiari criteri, un campione viene generalmente selezionato tramite un procedura casuale. Il \emph{campionamento casuale} consente allo scienziato di trarre conclusioni sulla popolazione e, soprattutto, di quantificare l'incertezza sui risultati. I campioni di un sondaggio sono esempi di campioni casuali, ma molti studi osservazionali non sono campionati casualmente. Possono essere \emph{campioni di convenienza}, come coorti di studenti in un unico istituto, che consistono di tutti gli studenti sottoposti ad un certo intervento psicologico in quell'istituto. Indipendentemente da come vengono ottenuti i campioni, il loro uso al fine di conoscere una popolazione target significa che i problemi di rappresentatività sono inevitabili e devono essere affrontati.

\hypertarget{variabili-e-costanti}{%
\section{Variabili e costanti}\label{variabili-e-costanti}}

Definiamo \emph{variabile statistica} la proprietà (o grandezza) che è
oggetto di studio nell'analisi dei dati. Una variabile è una proprietà
di un fenomeno che può essere espressa in più valori sia numerici sia
categoriali. Il termine ``variabile'' si contrappone al termine ``costante''
che descrive una proprietà invariante di tutte le unità statistiche.

Si dice \emph{modalità} ciascuna delle varianti con cui una variabile
statistica può presentarsi. Definiamo \emph{insieme delle modalità} di una
variabile statistica l'insieme \(M\) di tutte le possibili espressioni con
cui la variabile può manifestarsi. Le modalità osservate e facenti parte
del campione si chiamano \emph{dati} (si veda la
Tabella~\protect\hyperlink{tab:term_st_desc}{1.1}).

\begin{workedexample}
Supponiamo che il fenomeno studiato sia l'intelligenza. In uno studio, la popolazione potrebbe corrispondere all'insieme di tutti gli italiani adulti. La variabile considerata potrebbe essere il punteggio del test standardizzato WAIS-IV. Le modalità di tale variabile potrebbero essere \(112, 92, 121, \dots\). Tale variabile è di tipo quantitativo discreto.

\end{workedexample}

\begin{workedexample}
Supponiamo che il fenomeno studiato sia il compito Stroop. La popolazione potrebbe corrispondere all'insieme dei bambini dai 6 agli 8 anni. La variabile considerata potrebbe essere il reciproco dei tempi di reazione in secondi. Le modalità di tale variabile potrebbero essere \(1 / 2.35, 1/ 1.49, 1/2.93, \dots\). La variabile è di tipo quantitativo continuo.

\end{workedexample}

\begin{workedexample}
Supponiamo che il fenomeno studiato sia il disturbo di personalità. La popolazione potrebbe corrispondere all'insieme dei detenuti nelle carceri italiane. La variabile considerata potrebbe essere l'assessment del disturbo di personalità tramite interviste cliniche strutturate. Le modalità di tale variabile potrebbero essere i Cluster A, Cluster B, Cluster C descritti dal DSM-V. Tale variabile è di tipo qualitativo.

\end{workedexample}

\hypertarget{variabili-casuali}{%
\subsection{Variabili casuali}\label{variabili-casuali}}

Il termine \emph{variabile} usato nella statistica è equivalente al termine \emph{variabile casuale} usato nella teoria delle probabilità. Lo studio dei risultati degli interventi psicologici è lo studio delle variabili casuali che misurano questi risultati. Una variabile casuale cattura una caratteristica specifica degli individui nella popolazione e i suoi valori variano tipicamente tra gli individui. Ogni variabile casuale può assumere in teoria una gamma di valori sebbene, in pratica, osserviamo un valore specifico per ogni individuo. Quando faremo riferiremo alle variabili casuali considerate in termini generali useremo lettere maiuscole come \(X\) e \(Y\); quando faremo riferimento ai valori che una variabile casuale assume in determinate circostanze useremo lettere minuscole come \(x\) e \(y\).

\hypertarget{variabili-indipendenti-e-variabili-dipendenti}{%
\subsection{Variabili indipendenti e variabili dipendenti}\label{variabili-indipendenti-e-variabili-dipendenti}}

Un primo compito fondamentale in qualsiasi analisi dei dati è l'identificazione delle variabili dipendenti (\(Y\)) e delle variabili indipendenti (\(X\)). Le variabili dipendenti sono anche chiamate variabili di esito o di risposta e le variabili indipendenti sono anche chiamate predittori o covariate. Ad esempio, nell'analisi di regressione, che esamineremo in seguito, la domanda centrale è quella di capire come \(Y\) cambia al variare di \(X\). Più precisamente, la domanda che viene posta è: se il valore della variabile indipendente \(X\) cambia, qual è la conseguenza per la variabile dipendente \(Y\)? In parole povere, le variabili indipendenti e dipendenti sono analoghe a ``cause'' ed ``effetti'', laddove le virgolette usate qui sottolineano che questa è solo un'analogia e che la determinazione delle cause può avvenire soltanto mediante l'utilizzo di un appropriato disegno sperimentale e di un'adeguata analisi statistica.

Se una variabile è una variabile indipendente o dipendente dipende dalla domanda di ricerca. A volte può essere difficile decidere quale variabile è dipendente e quale è indipendente, in particolare quando siamo specificamente interessati ai rapporti di causa/effetto. Ad esempio, supponiamo di indagare l'associazione tra esercizio fisico e insonnia. Vi sono evidenze che l'esercizio fisico (fatto al momento giusto della giornata) può ridurre l'insonnia. Ma l'insonnia può anche ridurre la capacità di una persona di fare esercizio fisico. In questo caso, dunque, non è facile capire quale sia la causa e quale l'effetto, quale sia la variabile dipendente e quale la variabile indipendente. La possibilità di identificare il ruolo delle variabili (dipendente/indipendente) dipende dalla nostra comprensione del fenomeno in esame.

\begin{workedexample}
Uno psicologo convoca 120 studenti universitari per un test di memoria.
Prima di iniziare l'esperimento, a metà dei soggetti viene detto che si
tratta di un compito particolarmente difficile; agli altri soggetti non
viene data alcuna indicazione. Lo psicologo misura il punteggio nella
prova di memoria di ciascun soggetto.

In questo esperimento, la variabile indipendente è l'informazione sulla difficoltà della prova. La variabile indipendente viene manipolata dallo sperimentatore assegnando i soggetti (di solito in maniera causale) o alla condizione (modalità) ``informazione assegnata'' o ``informazione non data''. La
variabile dipendente è ciò che viene misurato nell'esperimento, ovvero
il punteggio nella prova di memoria di ciascun soggetto.

\end{workedexample}

\hypertarget{la-matrice-dei-dati}{%
\subsection{La matrice dei dati}\label{la-matrice-dei-dati}}

Le realizzazioni delle variabili esaminate in una rilevazione statistica
vengono organizzate in una \emph{matrice dei dati}. Le colonne della matrice
dei dati contengono gli insiemi dei dati individuali di ciascuna
variabile statistica considerata. Ogni riga della matrice contiene tutte
le informazioni relative alla stessa unità statistica. Una generica
matrice dei dati ha l'aspetto seguente:

\[
D_{m,n} = 
 \begin{pmatrix}
  \omega_1 & a_{1}   & b_{1}   & \cdots & x_{1} & y_{1}\\
  \omega_2 & a_{2}   & b_{2}   & \cdots & x_{2} & y_{2}\\
  \vdots   & \vdots  & \vdots  & \ddots & \vdots & \vdots  \\
 \omega_n  & a_{n}   & b_{n}   & \cdots & x_{n} & y_{n}
 \end{pmatrix}
 \]
\noindent
dove, nel caso presente, la prima colonna contiene il
nome delle unità statistiche, la seconda e la terza colonna si
riferiscono a due mutabili statistiche (variabili categoriali; \(A\) e
\(B\)) e ne presentano le modalità osservate nel campione mentre le ultime
due colonne si riferiscono a due variabili statistiche (\(X\) e \(Y\)) e ne
presentano le modalità osservate nel campione. Generalmente, tra le
unità statistiche \(\omega_i\) non esiste un ordine progressivo; l'indice
attribuito alle unità statistiche nella matrice dei dati si riferisce
semplicemente alla riga che esse occupano.

\hypertarget{parametri-e-modelli}{%
\section{Parametri e modelli}\label{parametri-e-modelli}}

Ogni variabile casuale ha una \emph{distribuzione} che descrive la probabilità che la variabile assuma qualsiasi valore in un dato intervallo.\footnote{In questo e nei successivi Paragrafi di questo Capitolo introduco gli obiettivi della \emph{data science} utilizzando una serie di concetti che saranno chiariti solo in seguito. Questa breve panoramica risulterà dunque solo in parte comprensibile ad una prima lettura e serve solo per definire la \emph{big picture} dei temi trattati in questo insegnamento. Il significato dei termini qui utilizzati sarà chiarito nei Capitoli successivi.} Senza ulteriori specificazioni, una distribuzione può fare riferimento a un'intera famiglia di distribuzioni. I parametri, tipicamente indicati con lettere greche come \(\mu\) e \(\alpha\), ci permettono di specificare di quale membro della famiglia stiamo parlando. Quindi, si può parlare di una variabile casuale con una distribuzione Normale, ma se viene specificata la media \(\mu\) = 100 e la varianza \(\sigma^2\) = 15, viene individuata una specifica distribuzione Normale -- nell'esempio, la distribuzione del quoziente di intelligenza.

I metodi statistici parametrici specificano la famiglia delle distribuzioni e quindi utilizzano i dati per individuare, stimando i parametri, una specifica distribuzione all'interno della famiglia di distribuzioni ipotizzata. Se \(f\) è la PDF di una variabile casuale \(Y\), l'interesse può concentrarsi sulla sua media e varianza. Nell'analisi di regressione, ad esempio, cerchiamo di spiegare come i parametri di \(f\) dipendano dalle covariate \(X\). Nella regressione lineare classica, assumiamo che \(Y\) abbia una distribuzione normale con media \(\mu = \E(Y)\), e stimiamo come \(\E(Y)\) dipenda da \(X\). Poiché molti esiti psicologici non seguono una distribuzione normale, verranno introdotte distribuzioni più appropriate per questi risultati. I metodi non parametrici, invece, non specificano una famiglia di distribuzioni per \(f\). In queste dispense faremo riferimento a metodi non parametrici quando discuteremo della statistica descrittiva.

Il termine \emph{modello} è onnipresente in statistica e nella \emph{data science}. Il modello statistico include le ipotesi e le specifiche matematiche relative alla distribuzione della variabile casuale di interesse. Il modello dipende dai dati e dalla domanda di ricerca, ma raramente è unico; nella maggior parte dei casi, esiste più di un modello che potrebbe ragionevolmente usato per affrontare la stessa domanda di ricerca e avendo a disposizione i dati osservati. Nella previsione delle aspettative future dei pazienti depressi che discuteremo in seguito \autocite{zetschefuture2019}, ad esempio, la specifica del modello include l'insieme delle covariate candidate, l'espressione matematica che collega i predittori con le aspattative future e qualsiasi ipotesi sulla distribuzione della variabile dipendente. La domanda di cosa costituisca un buon modello è una domanda su cui torneremo ripetutamente in questo insegnamento.

\hypertarget{effetto}{%
\section{Effetto}\label{effetto}}

L'\emph{effetto} è una qualche misura dei dati. Dipende dal tipo di dati e dal tipo di test statistico che si vuole utilizzare. Ad esempio, se viene lanciata una moneta 100 volte e esce testa 66 volte, l'effetto sarà 66/100. Diventa poi possibile confrontare l'effetto ottenuto con l'effetto nullo che ci si aspetterebbe da una moneta bilanciata (50/100), o con qualsiasi altro effetto che può essere scelto. La \emph{dimensione dell'effetto} si riferisce alla differenza tra l'effetto misurato nei dati e l'effetto nullo (di solito un valore che ci si aspetta di ottenere in base al caso soltanto).

\hypertarget{stima-e-inferenza}{%
\section{Stima e inferenza}\label{stima-e-inferenza}}

La stima è il processo mediante il quale il campione viene utilizzato per conoscere le proprietà di interesse della popolazione. La media campionaria è una stima naturale della media della popolazione e la mediana campionaria è una stima naturale della mediana della popolazione. Quando parliamo di stimare una proprietà della popolazione (a volte indicata come parametro della popolazione) o di stimare la distribuzione di una variabile casuale, stiamo parlando dell'utilizzo dei dati osservati per conoscere le proprietà di interesse della popolazione. L'inferenza statistica è il processo mediante il quale le stime campionarie vengono utilizzate per rispondere a domande di ricerca e per valutare specifiche ipotesi relative alla popolazione. Discuteremo le procedure bayesiane dell'inferenza nell'ultima parte di queste dispense.

\hypertarget{metodi-e-procedure-della-psicologia}{%
\section{Metodi e procedure della psicologia}\label{metodi-e-procedure-della-psicologia}}

Un modello psicologico di un qualche aspetto del comportamento umano o della mente ha le seguenti proprietà:

\begin{enumerate}
\def\labelenumi{\arabic{enumi}.}
\tightlist
\item
  descrive le caratteristiche del comportamento in questione,
\item
  formula predizioni sulle caratteristiche future del comportamento,
\item
  è sostenuto da evidenze empiriche,
\item
  deve essere falsificabile (ovvero, in linea di principio, deve
  potere fare delle predizioni su aspetti del fenomeno considerato che
  non sono ancora noti e che, se venissero indagati, potrebbero
  portare a rigettare il modello, se si dimostrassero incompatibili con
  esso).
\end{enumerate}

\noindent
L'analisi dei dati valuta un modello psicologico utilizzando strumenti statistici.

Questa dispensa è strutturata in maniera tale da rispecchiare la suddivisione tra i temi della misurazione, dell'analisi descrittiva e dell'inferenza. Nel prossimo Capitolo sarà affrontato il tema della misurazione e, nell'ultima parte della dispensa verrà discusso l'argomento più difficile, quello dell'inferenza. Prima di affrontare il secondo tema, l'analisi descrittiva dei dati, sarà necessario introdurre il linguaggio di programmazione statistica R (un'introduzione a R è fornita in Appendice). Inoltre, prima di potere discutere l'inferenza, dovranno essere introdotti i concetti di base della teoria delle probabilità, in quanto l'inferenza non è che l'applicazione della teoria delle probabilità all'analisi dei dati.


% Bibliography
%%%%%%%%%%%%%%%%%%%%%%%%%%%%%%%%%%%%%%%%%%%%%%%%%%%%%%%%%%

\backmatter
\SmallMargins

\printbibliography
\onecolumn


% Tables (of tables, of figures)
%%%%%%%%%%%%%%%%%%%%%%%%%%%%%%%%%%%%%%%%%%%%%%%%%%%%%%%%%%


\cleardoublepage
\LargeMargins
\listoffigures


% After-body (LaTeX code inclusion)
%%%%%%%%%%%%%%%%%%%%%%%%%%%%%%%%%%%%%%%%%%%%%%%%%%%%%%%%%%




% Back cover
%%%%%%%%%%%%%%%%%%%%%%%%%%%%%%%%%%%%%%%%%%%%%%%%%%%%%%%%%%%

% Even page, small margins, no running head, no page number.
\evenpage
\SmallMargins
\thispagestyle{empty}

\begin{normalsize}

\begin{description}

\selectlanguage{italian}
\item[Abstract]
This document contains the material of the lessons of Psicometria B000286 (2021/2022) aimed at students of the first year of the Degree Course in Psychological Sciences and Techniques of the University of Florence, Italy.
\item[Keywords]
Data science, Bayesian statistics.
~\\

\end{description}

\end{normalsize}


\end{document}
