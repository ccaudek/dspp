% Template adapted from https://github.com/jgm/pandoc-templates/blob/master/default.latex
% To be used with XeLaTex in memoiR
%%%%%%%%%%%%%%%%%%%%%%%%%%%%%%%%%%%%%%%%%%%%%%%%%%%%%%%%%%%%%%%%%%%%%%%%%%%%%%%%%%%%%%%%%

% Options for packages loaded elsewhere
\PassOptionsToPackage{unicode=true}{hyperref}
\PassOptionsToPackage{hyphens}{url}
\PassOptionsToPackage{dvipsnames,svgnames*,x11names*}{xcolor}
% Right to left support


\documentclass[
  10pt,
  italian,
  a4paper,
  extrafontsizes,onecolumn,openright
  ]{memoir}

% Double (or whatever) spacing

% Math
\usepackage{amssymb, amsmath}
% mathspec: arbitrary math fonts
\usepackage{unicode-math}
\defaultfontfeatures{Scale=MatchLowercase}
\defaultfontfeatures[\rmfamily]{Ligatures=TeX,Scale=1}

% Fonts
\usepackage{lmodern}
\usepackage{fontspec}

% Main font
% Specific sanserif font
% Specific monotype font
\setmonofont[Scale=0.75]{Operator Mono SSm Lig Book}
% Specific math font
% Chinese, Japanese, Corean fonts

% Use upquote for straight quotes in verbatim environments
\usepackage{upquote}
% Use microtype
\usepackage[]{microtype}
\UseMicrotypeSet[protrusion]{basicmath} % disable protrusion for tt fonts

% Verbatim in note

% Color links
\usepackage{xcolor}

% Strikeout

% Necessary for code chunks
\usepackage{color}
\usepackage{fancyvrb}
\newcommand{\VerbBar}{|}
\newcommand{\VERB}{\Verb[commandchars=\\\{\}]}
\DefineVerbatimEnvironment{Highlighting}{Verbatim}{commandchars=\\\{\}}
% Add ',fontsize=\small' for more characters per line
\usepackage{framed}
\definecolor{shadecolor}{RGB}{248,248,248}
\newenvironment{Shaded}{\begin{snugshade}}{\end{snugshade}}
\newcommand{\AlertTok}[1]{\textcolor[rgb]{0.94,0.16,0.16}{#1}}
\newcommand{\AnnotationTok}[1]{\textcolor[rgb]{0.56,0.35,0.01}{\textbf{\textit{#1}}}}
\newcommand{\AttributeTok}[1]{\textcolor[rgb]{0.77,0.63,0.00}{#1}}
\newcommand{\BaseNTok}[1]{\textcolor[rgb]{0.00,0.00,0.81}{#1}}
\newcommand{\BuiltInTok}[1]{#1}
\newcommand{\CharTok}[1]{\textcolor[rgb]{0.31,0.60,0.02}{#1}}
\newcommand{\CommentTok}[1]{\textcolor[rgb]{0.56,0.35,0.01}{\textit{#1}}}
\newcommand{\CommentVarTok}[1]{\textcolor[rgb]{0.56,0.35,0.01}{\textbf{\textit{#1}}}}
\newcommand{\ConstantTok}[1]{\textcolor[rgb]{0.00,0.00,0.00}{#1}}
\newcommand{\ControlFlowTok}[1]{\textcolor[rgb]{0.13,0.29,0.53}{\textbf{#1}}}
\newcommand{\DataTypeTok}[1]{\textcolor[rgb]{0.13,0.29,0.53}{#1}}
\newcommand{\DecValTok}[1]{\textcolor[rgb]{0.00,0.00,0.81}{#1}}
\newcommand{\DocumentationTok}[1]{\textcolor[rgb]{0.56,0.35,0.01}{\textbf{\textit{#1}}}}
\newcommand{\ErrorTok}[1]{\textcolor[rgb]{0.64,0.00,0.00}{\textbf{#1}}}
\newcommand{\ExtensionTok}[1]{#1}
\newcommand{\FloatTok}[1]{\textcolor[rgb]{0.00,0.00,0.81}{#1}}
\newcommand{\FunctionTok}[1]{\textcolor[rgb]{0.00,0.00,0.00}{#1}}
\newcommand{\ImportTok}[1]{#1}
\newcommand{\InformationTok}[1]{\textcolor[rgb]{0.56,0.35,0.01}{\textbf{\textit{#1}}}}
\newcommand{\KeywordTok}[1]{\textcolor[rgb]{0.13,0.29,0.53}{\textbf{#1}}}
\newcommand{\NormalTok}[1]{#1}
\newcommand{\OperatorTok}[1]{\textcolor[rgb]{0.81,0.36,0.00}{\textbf{#1}}}
\newcommand{\OtherTok}[1]{\textcolor[rgb]{0.56,0.35,0.01}{#1}}
\newcommand{\PreprocessorTok}[1]{\textcolor[rgb]{0.56,0.35,0.01}{\textit{#1}}}
\newcommand{\RegionMarkerTok}[1]{#1}
\newcommand{\SpecialCharTok}[1]{\textcolor[rgb]{0.00,0.00,0.00}{#1}}
\newcommand{\SpecialStringTok}[1]{\textcolor[rgb]{0.31,0.60,0.02}{#1}}
\newcommand{\StringTok}[1]{\textcolor[rgb]{0.31,0.60,0.02}{#1}}
\newcommand{\VariableTok}[1]{\textcolor[rgb]{0.00,0.00,0.00}{#1}}
\newcommand{\VerbatimStringTok}[1]{\textcolor[rgb]{0.31,0.60,0.02}{#1}}
\newcommand{\WarningTok}[1]{\textcolor[rgb]{0.56,0.35,0.01}{\textbf{\textit{#1}}}}

% Listings package

% Tables
\usepackage{longtable,booktabs,tabu}
% Fix footnotes in tables (requires footnote package)
\IfFileExists{footnote.sty}{\usepackage{footnote}\makesavenoteenv{longtable}}{}

% Graphics
\usepackage{graphicx,grffile}
\graphicspath{{images/}}
\makeatletter
\def\maxwidth{\ifdim\Gin@nat@width>\linewidth\linewidth\else\Gin@nat@width\fi}
\def\maxheight{\ifdim\Gin@nat@height>\textheight\textheight\else\Gin@nat@height\fi}
\makeatother
% Scale images if necessary, so that they will not overflow the page
% margins by default, and it is still possible to overwrite the defaults
% using explicit options in \includegraphics[width, height, ...]{}
\setkeys{Gin}{width=\maxwidth,height=\maxheight,keepaspectratio}

% Prevent overfull lines
\setlength{\emergencystretch}{3em}  
\providecommand{\tightlist}{%
  \setlength{\itemsep}{0pt}\setlength{\parskip}{0pt}}

% Number sections for memoir (secnumdepth counter is ignored)
\setsecnumdepth{section}

% Set default figure placement to htbp
\makeatletter
\def\fps@figure{htbp}
\makeatother

% Spacing in lists
\usepackage{enumitem}

% Polyglossia
\usepackage{polyglossia}
\setmainlanguage{it}
\setotherlanguage{en-US}

% BibLaTeX
\usepackage[backend=biber,style=authoryear-ibid,isbn=false,backref=true,giveninits=true,uniquename=init,maxcitenames=2,maxbibnames=150,sorting=nyt,sortcites=false,style=apa]{biblatex}
\addbibresource{refs.bib}

% cslreferences environment required by pandoc > 2.7



%%%%%%%%%%%%%%%%%%%%%%%%%%%%%%%%%%%%%%%%%%%%%%%%%%%%%%%%%%
% memoiR format

% Chapter Summary environment 
\usepackage[tikz]{bclogo}
\newenvironment{Summary}
  {\begin{bclogo}[logo=\bctrombone, noborder=true, couleur=lightgray!50]{In breve}\parindent0pt}
  {\end{bclogo}}
% Syntax:
%
%```{block, type='Summary'}
% Deliver message here.
% ```

% scriptsize code 
\let\oldverbatim\verbatim
\def\verbatim{\oldverbatim\scriptsize}
% Applies to code blocks and R code results
% code chunk options size='scriptsize' applies only to R code and results
% if the code chunk sets a different size, \def\verbatim{...} is prioritary for code results 


% Layout
%%%%%%%%%%%%%%%%%%%%%%%%%%%%%%%%%%%%%%%%%%%%%%%%%%%%%%%%%%

% Based on memoir, style companion
\newcommand{\MemoirChapStyle}{daleif1}
\newcommand{\MemoirPageStyle}{Ruled}

% Space between paragraphs
\usepackage{parskip}
  \abnormalparskip{3pt}

% Adjust margin paragraphs vertical position
\usepackage{marginfix}


% Margins
%%%%%%%%%%%%%%%%%%%%%%%%%%%%%%%%%%%%%%%

% allow use of '-',+','/' ans '*' to make simple length computation
\usepackage{calc}

% Full-width figures utilities
\newlength\widthw % full width
\newlength{\rf}
\newcommand*{\definesHSpace}{
  \strictpagecheck % slower but efficient detection of odd/even pages
  \checkoddpage
  \ifoddpage
  \setlength{\rf}{0mm}
  \else
  \setlength{\rf}{\marginparsep+\marginparwidth}
  \fi
}

\makeatletter
% 1" margins for the front matter.
\newcommand*{\SmallMargins}{
  \setlrmarginsandblock{1.5in}{1.5in}{*}
  \setmarginnotes{0.1in}{0.1in}{0.1in}
 \setulmarginsandblock{1.5in}{1in}{*}
  \checkandfixthelayout
  \ch@ngetext
  \clearpage
  \setlength{\widthw}{\textwidth+\marginparsep+\marginparwidth}
  \footnotesatfoot
  \chapterstyle{\MemoirChapStyle}  % Chapter and page styles must be recalled
  \pagestyle{\MemoirPageStyle}
}

% 3" outer margin for the main matter
\newcommand{\LargeMargins}{\SmallMargins}
\makeatother

% Figure captions and footnotes in outer margins


% Main title page with filigrane
%%%%%%%%%%%%%%%%%%%%%%%%%%%%%%%%%%%%%%%%%%%%%%%%%%%%%%%%%%

% Text blocks
\usepackage[absolute,overlay]{textpos}
  \setlength{\TPHorizModule}{1mm}
  \setlength{\TPVertModule}{1mm}

\newcommand{\MainTitlePage}[2]{
  \SmallMargins % Margins
  \pagestyle{empty} % No header/footer
  \textblockorigin{\stockwidth-\paperwidth-\trimedge}{\trimtop} % recto
  \begin{textblock*}{2mm}(\spinemargin/2,\uppermargin/2)
    \rule{1pt}{\paperheight-\uppermargin}
  \end{textblock*}
  \begin{textblock*}{\paperwidth*2/3}(\paperwidth/5, \paperheight/5)
    \flushright
    \begin{Spacing}{3}
      {\fontfamily{qtm}\selectfont\fontsize{45}{45}\selectfont\textsc{\thetitle}}
    \end{Spacing}
  \end{textblock*}
    \begin{textblock*}{\paperwidth*2/3}(\paperwidth/5, \paperheight/2)
    \flushright
    {\fontfamily{qtm}\huge\theauthor}
  \end{textblock*}
    \begin{textblock*}{\paperwidth*2/3}[0, 1](\spinemargin, \uppermargin+\textheight)
    \normalfont\thedate
  \end{textblock*}
  ~\\ % Print a character or the page will not exist
  \newpage
  \textblockorigin{\trimedge}{\trimtop} % verso
  \begin{textblock*}{\textwidth}(\paperwidth-\spinemargin-\textwidth, \uppermargin)
    #1
  \end{textblock*}
  \begin{textblock*}{\textwidth}[0,1](\paperwidth-\spinemargin-\textwidth, \uppermargin+\textheight+\footskip)
    \centering
    \includegraphics[width=\paperwidth/4]{logo}\\ \bigskip
    #2
  \end{textblock*}
  ~\\ % Print a character or the page will not exist
  \newpage
}

% Clear page and open an even one (\clearpage opens an odd one)
\newcommand{\evenpage}{
  \clearpage
  \strictpagecheck % slower but efficient detection of odd/even pages
  \checkoddpage
  \ifoddpage
    \thispagestyle{empty}
    ~\\ % Print a character or the page will not exist
    \newpage
  \else
    % do nothing
  \fi
}


%% PDF title page to insert
%%%%%%%%%%%%%%%%%%%%%%%%%%%%%%%%%%%%%%%%%%%%%%%%%%%%%%%%%%



%% Bibliography
%%%%%%%%%%%%%%%%%%%%%%%%%%%%%%%%%%%%%%%%%%%%%%%%%%%%%%%%%%

\usepackage[strict,autostyle]{csquotes}
% Repeated citation as author-year-title instead of author-title (modification of footcite:note in verbose-inote.cbx)

%% Table of Contents
%%%%%%%%%%%%%%%%%%%%%%%%%%%%%%%%%%%%%%%%%%%%%%%%%%%%%%%%%%

% fix the typesetting of the part number
\renewcommand\partnumberlinebox[2]{#2\ ---\ }


% Fonts
%%%%%%%%%%%%%%%%%%%%%%%%%%%%%%%%%%%%%%%%%%%%%%%%%%%%%%%%%%


% Hyperref comes last
%%%%%%%%%%%%%%%%%%%%%%%%%%%%%%%%%%%%%%%%%%%%%%%%%%%%%%%%%%

\usepackage{hyperref}
\hypersetup{
  pdftitle={Psicometria},
  pdfauthor={Corrado Caudek},
  colorlinks=true,
  linkcolor=Maroon,
  citecolor=Blue,
  urlcolor=Blue,
  breaklinks=true}

% Don't use monospace font for urls
\urlstyle{same}


% Title, author, date from YAML to LaTeX
%%%%%%%%%%%%%%%%%%%%%%%%%%%%%%%%%%%%%%%%%%%%%%%%%%%%%%%%%%

\title{Psicometria}

\author{Corrado Caudek}

\date{2021-12-22}


% Include headers (preamble.tex) here
%%%%%%%%%%%%%%%%%%%%%%%%%%%%%%%%%%%%%%%%%%%%%%%%%%%%%%%%%%
% Add LaTeX code into the preamble of the document here
\hyphenation{bio-di-ver-si-ty sap-lings}


%%%%%%%%%%%%%%%%%%%%%%%%%%%%%%%%%%%%%%%%%%%%%%%%%%%%%%%%%%%%%%%%%%%%%%%%%
% memoiR dalef3 chapter style 
% https://ctan.crest.fr/tex-archive/info/latex-samples/MemoirChapStyles/MemoirChapStyles.pdf
\usepackage{soul}
\definecolor{nicered}{rgb}{.647,.129,.149}

\makeatletter
\makechapterstyle{pedersen}
\makeatother

%\makeatletter
%\newlength\dlf@normtxtw
%\setlength\dlf@normtxtw{\textwidth}
%\def\myhelvetfont{\def\sfdefault{mdput}}
%\newsavebox{\feline@chapter}
%\newcommand\feline@chapter@marker[1][4cm]{%
%  \sbox\feline@chapter{%
%    \resizebox{!}{#1}{\fboxsep=1pt%
%	  \colorbox{nicered}{\color{white}\bfseries\sffamily\thechapter}%
%	}}%
%  \rotatebox{90}{%
%    \resizebox{%
%	  \heightof{\usebox{\feline@chapter}}+\depthof{\usebox{\feline@chapter}}}%
%	{!}{\scshape\so\@chapapp}}\quad%
%  \raisebox{\depthof{\usebox{\feline@chapter}}}{\usebox{\feline@chapter}}%
% }
%\newcommand\feline@chm[1][4cm]{%
%  \sbox\feline@chapter{\feline@chapter@marker[#1]}%
%  \makebox[0pt][l]{% aka \rlap
%    \makebox[1cm][r]{\usebox\feline@chapter}%
%  }}
%\makechapterstyle{pedersen}{ %daleif1
%
%  \renewcommand\chapnamefont{\normalfont\Large\scshape\raggedleft\so}
%  
%  % I changed this!!
%  %\renewcommand\chaptitlefont{\normalfont\huge\bfseries\scshape\color{nicered}}
%  \renewcommand\chaptitlefont{\normalfont\huge\fontencoding{T1}\fontfamily{phv}\selectfont\color{nicered}}
%    
%  \renewcommand\chapternamenum{}
%  \renewcommand\printchaptername{}
%  \renewcommand\printchapternum{\null\hfill\feline@chm[2.5cm]\par}
%  \renewcommand\afterchapternum{\par\vskip\midchapskip}
%  \renewcommand\printchaptertitle[1]{\chaptitlefont\raggedleft ##1\par}
%}
%\makeatother

\DeclareMathOperator{\Var}{Var} % Define variance operator
\DeclareMathOperator{\SD}{SD} % Define sd operator
\DeclareMathOperator{\Cov}{Cov} % Define covariance operator
\DeclareMathOperator{\Corr}{Corr} % Define correlation operator
\DeclareMathOperator{\Me}{Me} % Define mediane operator
\DeclareMathOperator{\Mo}{Mo} % Define mode operator
\DeclareMathOperator{\Bin}{Bin} % Define binomial operator
\DeclareMathOperator{\Bernoulli}{Bernoulli} % Define Bernoulli operator
\DeclareMathOperator{\Poi}{Poi} % Define Poisson operator
\DeclareMathOperator{\Uniform}{Uniform} % Define Uniform operator
\DeclareMathOperator{\Cauchy}{Cauchy} % Define Cauchy operator
\DeclareMathOperator{\elpd}{elpd} % Define elpd operator
\DeclareMathOperator{\lppd}{lppd} % Define lppd operator
\DeclareMathOperator{\LOO}{LOO} % Define LOO operator
\DeclareMathOperator{\Ber}{\mathscr{B}} % Define Bernoulli operator
\DeclareMathOperator{\B}{B} % beta function
% \mbox{B}(a, b) % beta function
% \mbox{Beta}(a, b) % beta distribution
\newcommand{\R}{\textsf{R}} % Define R programming language symbol
\newcommand{\E}{\mathbb{E}} % Define expected value operator
\newcommand{\Real}{\mathbb{R}} % Define real number operator
\newcommand{\Prob}{\mathscr{P}}
\DeclareMathOperator{\argmin}{arg\,min} % thin space, limits on side in displays
\DeclareMathOperator{\argmax}{arg\,max} % no space, limits on side in displays

\raggedbottom % allow variable (ragged) site heights
\frenchspacing

\usepackage[
  labelfont=bf, 
  font={small, it} 
]{caption} 
\usepackage{upquote} % print correct quotes in verbatim-environments
\usepackage{empheq} 
\usepackage{xfrac}
\usepackage{lstbayes}
\usepackage{awesomebox}


% Introduction to Modern statistics ------------------------------------------------
% https://github.com/OpenIntroStat/ims/blob/main/latex/ims-style.tex

\usepackage[framemethod=tikz]{mdframed} 
\usepackage{helvet} 
\usepackage{xcolor}


\definecolor{oiB}{HTML}{000000}            % COL["blue","full"]
\definecolor{oiLB}{HTML}{e0e0e0}           % lighter version of oiB

\definecolor{oiY}{HTML}{000000}            % COL["yellow","full"]
\definecolor{oiLY}{HTML}{e0e0e0}           % lighter version of oiY

\definecolor{oiR}{HTML}{000000}            % COL["red","full"]
\definecolor{oiLR}{HTML}{e0e0e0}           % lighter version of oiR

\definecolor{oiGray}{HTML}{808080}         % COL["gray","full"]
\definecolor{oiLGray}{HTML}{f8f8f8}        % lighter version of oiR

\definecolor{oiGB}{rgb}{0.5,0.5,.5}        % from OS4 - for footnotes


% Helper environments ------------------------------------------------------------

% mdframedwithfootChapterintro: for chapterintro box

\newenvironment{mdframedwithfootChapterintro}
{   
    \savenotes
    \begin{mdframed}[%
    topline=true, bottomline=true, linecolor=oiB, linewidth=1.4pt,
    rightline=false, leftline=false,
    backgroundcolor=oiLB]
    %\stepcounter{footnote} % don't increment footnote counter
    \renewcommand{\thempfootnote}{\arabic{footnote}}
    }
{
    \end{mdframed}
    \spewnotes
}


% mdframedwithfootGPWE: for guidedpractice and workedexample

\newenvironment{mdframedwithfootGPWE}
{   
    \savenotes
    \begin{mdframed}[%
    topline=true, bottomline=true, linecolor=oiB, linewidth=0.5pt,
    rightline=false, leftline=false,
    backgroundcolor=oiLGray]
    %\stepcounter{footnote}
    \renewcommand{\thempfootnote}{\arabic{footnote}}
    }
{
    \end{mdframed}
    \spewnotes
}


% mdframedwithfootImportant: for important

\newenvironment{mdframedwithfootImportant}
{   
    \savenotes
    \begin{mdframed}[%
    topline=true, bottomline=true, linecolor=oiR, linewidth=0.5pt,
    rightline=false, leftline=false,
    backgroundcolor=oiLGray]
    %\stepcounter{footnote}
    \renewcommand{\thempfootnote}{\arabic{footnote}}
    }
{
    \end{mdframed}
    \spewnotes
}


% mdframedwithfootTip: for tip, data, and pronunciation

\newenvironment{mdframedwithfootTipDataPro}
{   
    \savenotes
    \begin{mdframed}[%
    topline=true, bottomline=true, linecolor=oiGray, linewidth=0.5pt,
    rightline=false, leftline=false,
    backgroundcolor=oiLGray]
    %\stepcounter{footnote}
    \renewcommand{\thempfootnote}{\arabic{footnote}}
    }
{
    \end{mdframed}
    \spewnotes
}


% Custom environments/boxes -------------------------------------------------------

% chapterintro

\newenvironment{chapterintro}{
\vspace{4mm}
\begin{mdframedwithfootChapterintro}
\begin{minipage}[t]{0.10\textwidth}
{$\:$ \\ \setkeys{Gin}{width=2.5em,keepaspectratio}\includegraphics{images/_icons/chapterintro.png}}
\end{minipage}
\hfill
\begin{minipage}[t]{0.90\textwidth}
\setlength{\parskip}{1em}
\large
}{\end{minipage}
\end{mdframedwithfootChapterintro}
\vspace{4mm}
}

% guidedpractice

\newenvironment{guidedpractice}{
\vspace{4mm}
\begin{mdframedwithfootGPWE}
\begin{minipage}[t]{0.10\textwidth}
{$\:$ \\ \setkeys{Gin}{width=2.5em,keepaspectratio}\includegraphics{images/_icons/guided-practice.png}}
\end{minipage}
\hfill
\begin{minipage}[t]{0.90\textwidth}
\vspace{-2mm}
\setlength{\parskip}{1em}
\noindent\textbf{\color{oiB}\small\fontencoding{T1}\fontfamily{phv}\selectfont{\MakeUppercase{Pratica guidata}}} $\:$ \\ \\
}{\end{minipage}
\end{mdframedwithfootGPWE}
\vspace{4mm}
}


% workedexample

\newenvironment{workedexample}{
    \let\oldrule\rule
    \renewcommand{\rule}[2]{\vspace{-2mm}\oldrule{##1}{##2}\vspace{-2mm}}
\vspace{4mm}
\begin{mdframedwithfootGPWE}
\begin{minipage}[t]{0.10\textwidth}
{$\:$ \\ \setkeys{Gin}{width=2.5em,keepaspectratio}\includegraphics{images/_icons/worked-example.png}}
\end{minipage}
\hfill
\begin{minipage}[t]{0.90\textwidth}
\vspace{-2mm}
\setlength{\parskip}{1em}
\noindent\textbf{\color{oiB}\small\fontencoding{T1}\fontfamily{phv}\selectfont{\MakeUppercase{Esempio}}} $\:$ \\ \\
}{\end{minipage}
\end{mdframedwithfootGPWE}
\vspace{4mm}
}


% important

\newenvironment{important}{
    \let\oldtextbf\textbf
    \renewcommand{\textbf}[1]{{\textcolor{oiR}{\oldtextbf{##1}}}}
\vspace{4mm}
\begin{mdframedwithfootImportant}
\begin{minipage}[t]{0.10\textwidth}
{$\:$ \\ \setkeys{Gin}{width=2.5em,keepaspectratio}\includegraphics{images/_icons/important.png}}
\end{minipage}
\hfill
\begin{minipage}[t]{0.90\textwidth}
\vspace{-2mm}
\setlength{\parskip}{1em}
}{\end{minipage}
\end{mdframedwithfootImportant}
\vspace{4mm}
}

% tip

\newenvironment{tip}{
\vspace{4mm}
\begin{mdframedwithfootTipDataPro}
\begin{minipage}[t]{0.10\textwidth}
{$\:$ \\ \setkeys{Gin}{width=2em,keepaspectratio}\includegraphics{images/_icons/tip.png}}
\end{minipage}
\hfill
\begin{minipage}[t]{0.90\textwidth}
\vspace{-2mm}
\setlength{\parskip}{1em}
}{\end{minipage}
\end{mdframedwithfootTipDataPro}
\vspace{4mm}
}

% data

\newenvironment{data}{
\vspace{4mm}
\begin{mdframedwithfootTipDataPro}
\begin{minipage}[t]{0.10\textwidth}
{$\:$ \\ \setkeys{Gin}{width=2em,keepaspectratio}\includegraphics{images/_icons/data.png}}
\end{minipage}
\hfill
\begin{minipage}[t]{0.90\textwidth}
\vspace{-2mm}
\setlength{\parskip}{1em}
}{\end{minipage}
\end{mdframedwithfootTipDataPro}
\vspace{4mm}
}

%\usepackage{titlesec}
%\titleformat{\chapter}[display]
%  {\normalfont\sffamily\huge\bfseries\color{blue}}
%  {\chaptertitlename\ \thechapter}{20pt}{\Huge}
%\titleformat{\section}
%  {\normalfont\sffamily\Large\bfseries\color{cyan}}
%  {\thesection}{1em}{}

%%%%%%%%%%%%%%%%%%%%%%%%%%%%


\usepackage{booktabs}
\usepackage{longtable}
\usepackage{array}
\usepackage{multirow}
\usepackage{wrapfig}
\usepackage{float}
\usepackage{colortbl}
\usepackage{pdflscape}
\usepackage{tabu}
\usepackage{threeparttable}
\usepackage{threeparttablex}
\usepackage[normalem]{ulem}
\usepackage{makecell}
\usepackage{xcolor}


% End of preamble
%%%%%%%%%%%%%%%%%%%%%%%%%%%%%%%%%%%%%%%%%%%%%%%%%%%%%%%%%%


\usepackage{amsthm}
\newtheorem{theorem}{Teorema}[chapter]
\newtheorem{lemma}{Lemma}[chapter]
\newtheorem{corollary}{Corollario}[chapter]
\newtheorem{proposition}{Proposizione}[chapter]
\newtheorem{conjecture}{Congettura}[chapter]
\theoremstyle{definition}
\newtheorem{definition}{Definizione}[chapter]
\theoremstyle{definition}
\newtheorem{example}{Esempio}[chapter]
\theoremstyle{definition}
\newtheorem{exercise}{Esercizio}[chapter]
\theoremstyle{definition}
\newtheorem{hypothesis}{Hypothesis}[chapter]
\theoremstyle{remark}
\newtheorem*{remark}{Osservazione}
\newtheorem*{solution}{Soluzione}
\begin{document}
\frontmatter

% Title page
%%%%%%%%%%%%%%%%%%%%%%%%%%%%%%%%%%%%%%%%%%%%%%%%%%%%%%%%%%


\MainTitlePage{Questo documento è stato realizzato con:

\begin{itemize}
  \item \LaTeX\; e la classe memoir (\url{http://www.ctan.org/pkg/memoir});
  \item $\R$ (\url{http://www.r-project.org/}) e RStudio (\url{http://www.rstudio.com/});
  \item bookdown (\url{http://bookdown.org/}) e memoiR (\url{https://ericmarcon.github.io/memoiR/}).
  \end{itemize}}{Nel blog della mia pagina personale sono forniti alcuni approfondimenti degli argomenti qui trattati.
\url{https://ccaudek.github.io/caudeklab/}}


% Before Body
%%%%%%%%%%%%%%%%%%%%%%%%%%%%%%%%%%%%%%%%%%%%%%%%%%%%%%%%%%





% Contents
%%%%%%%%%%%%%%%%%%%%%%%%%%%%%%%%%%%%%%%%%%%%%%%%%%%%%%%%%%

\LargeMargins
{
\hypersetup{linkcolor=}
\setcounter{tocdepth}{2}
\tableofcontents
}


% Body
%%%%%%%%%%%%%%%%%%%%%%%%%%%%%%%%%%%%%%%%%%%%%%%%%%%%%%%%%%

\LargeMargins
\scriptsize

\normalsize

\chapter*{}

\vfill

\scriptsize

\normalsize

\scriptsize

Copyright \(\copyright\) 2022.

\normalsize

Data della versione presente: Dicembre 22, 2021.

\hypertarget{prefazione}{%
\chapter{Prefazione}\label{prefazione}}

\textbf{Data Science per psicologi} contiene il materiale delle lezioni dell'insegnamento di \emph{Psicometria B000286} (A.A. 2021/2022) rivolto agli studenti del primo anno del Corso di Laurea in Scienze e Tecniche Psicologiche dell'Università degli Studi di Firenze.

L'insegnamento di Psicometria si propone di fornire agli studenti un'introduzione all'analisi dei dati in psicologia.
Le conoscenze/competenze che verranno sviluppate in questo insegnamento sono quelle della \emph{Data science}, ovvero le conoscenze/competenze che si pongono all'intersezione tra statistica (ovvero, richiedono la capacità di comprendere teoremi statistici) e informatica (ovvero, richiedono la capacità di sapere utilizzare un software).

\hypertarget{la-psicologia-e-la-data-science}{%
\section*{La psicologia e la Data Science}\label{la-psicologia-e-la-data-science}}
\addcontentsline{toc}{section}{La psicologia e la Data Science}

\begin{quote}
\emph{It's worth noting, before getting started, that this material is hard. If you find yourself confused at any point, you are normal. Any sense of confusion you feel is just your brain correctly calibrating to the subject matter. Over time, confusion is replaced by comprehension {[}\ldots{]}}

\hfill --- Richard McElreath
\end{quote}

Sembra sensato spendere due parole su un tema che è importante per gli studenti: quello indicato dal titolo di questo Capitolo. È ovvio che agli studenti di psicologia la statistica non piace. Se piacesse, forse studierebbero Data Science e non psicologia; ma non lo fanno. Di conseguenza, gli studenti di psicologia si chiedono: ``perché dobbiamo perdere tanto tempo a studiare queste cose quando in realtà quello che ci interessa è tutt'altro?'\,' Questa è una bella domanda.

C'è una ragione molto semplice che dovrebbe farci capire perché la Data Science è così importante per la psicologia. Infatti, a ben pensarci, la psicologia è una disciplina intrinsecamente statistica, se per statistica intendiamo quella disciplina che studia la variazione delle caratteristiche degli individui nella popolazione. La psicologia studia \emph{gli individui} ed è proprio la variabilità inter- e intra-individuale ciò che vogliamo descrivere e, in certi casi, predire. In questo senso, la psicologia è molto diversa dall'ingegneria, per esempio. Le proprietà di un determinato ponte sotto certe condizioni, ad esempio, sono molto simili a quelle di un altro ponte, sotto le medesime condizioni. Quindi, per un ingegnere la statistica è poco importante: le proprietà dei materiali sono unicamente dipendenti dalla loro composizione e restano costanti. Ma lo stesso non può dirsi degli individui: ogni individuo è unico e cambia nel tempo. E le variazioni tra gli individui, e di un individuo nel tempo, sono l'oggetto di studio proprio della psicologia: è dunque chiaro che i problemi che la psicologia si pone sono molto diversi da quelli affrontati, per esempio, dagli ingegneri. Questa è la ragione per cui abbiamo tanto bisogno della \emph{data science} in psicologia: perché la \emph{data science} ci consente di descrivere la variazione e il cambiamento. E queste sono appunto le caratteristiche di base dei fenomeni psicologici.

Sono sicuro che, leggendo queste righe, a molti studenti sarà venuta in mente la seguente domanda: perché non chiediamo a qualche esperto di fare il ``lavoro sporco'' (ovvero le analisi statistiche) per noi, mentre noi (gli psicologi) ci occupiamo solo di ciò che ci interessa, ovvero dei problemi psicologici slegati dai dettagli ``tecnici'' della \emph{data science}?
La risposta a questa domanda è che non è possibile progettare uno studio psicologico sensato senza avere almeno una comprensione rudimentale della \emph{data science}. Le tematiche della \emph{data science} non possono essere ignorate né dai ricercatori in psicologia né da coloro che svolgono la professione di psicologo al di fuori dell'Università. Infatti, anche i professionisti al di fuori dall'università non possono fare a meno di leggere la letteratura psicologica più recente: il continuo aggiornamento delle conoscenze è infatti richiesto dalla deontologia della professione. Ma per potere fare questo è necessario conoscere un bel po' di \emph{data science}! Basta aprire a caso una rivista specialistica di psicologia per rendersi conto di quanto ciò sia vero: gli articoli che riportano i risultati delle ricerche psicologiche sono zeppi di analisi statistiche e di modelli formali. E la comprensione della letteratura psicologica rappresenta un requisito minimo nel bagaglio professionale dello psicologo.

Le considerazioni precedenti cercano di chiarire il seguente punto: la \emph{data science} non è qualcosa da studiare a malincuore, in un singolo insegnamento universitario, per poi poterla tranquillamente dimenticare. Nel bene e nel male, gli psicologi usano gli strumenti della \emph{data science} in tantissimi ambiti della loro attività professionale: in particolare quando costruiscono, somministrano e interpretano i test psicometrici. È dunque chiaro che possedere delle solide basi di \emph{data science} è un tassello imprescindibile del bagaglio professionale dello psicologo. In questo insegnamento verrano trattati i temi base della \emph{data science} e verrà adottato un punto di vista bayesiano, che corrisponde all'approccio più recente e sempre più diffuso in psicologia.

\hypertarget{come-studiare}{%
\section*{Come studiare}\label{come-studiare}}
\addcontentsline{toc}{section}{Come studiare}

\begin{quote}
\emph{I know quite certainly that I myself have no special talent. Curiosity, obsession and dogged endurance, combined with self-criticism, have brought me to my ideas.}

\hfill --- Albert Einstein
\end{quote}

Il giusto metodo di studio per prepararsi all'esame di Psicometria è quello di seguire attivamente le lezioni, assimilare i concetti via via che essi vengono presentati e verificare in autonomia le procedure presentate a lezione. Incoraggio gli studenti a farmi domande per chiarire ciò che non è stato capito appieno. Incoraggio gli studenti a utilizzare i forum attivi su Moodle e, soprattutto, a svolgere gli esercizi proposti su Moodle. I problemi forniti su Moodle rappresentano il livello di difficoltà richiesto per superare l'esame e consentono allo studente di comprendere se le competenze sviluppate fino a quel punto sono sufficienti rispetto alle richieste dell'esame.

La prima fase dello studio, che è sicuramente individuale, è quella in cui è necessario acquisire le conoscenze teoriche relative ai problemi che saranno presentati all'esame. La seconda fase di studio, che può essere facilitata da scambi con altri e da incontri di gruppo, porta ad acquisire la capacità di applicare le conoscenze: è necessario capire come usare un software (\(\textsf{R}\)) per applicare i concetti statistici alla specifica situazione del problema che si vuole risolvere. Le due fasi non sono però separate: il saper fare molto spesso ci aiuta a capire meglio.

\hypertarget{sviluppare-un-metodo-di-studio-efficace}{%
\section*{Sviluppare un metodo di studio efficace}\label{sviluppare-un-metodo-di-studio-efficace}}
\addcontentsline{toc}{section}{Sviluppare un metodo di studio efficace}

\begin{quote}
\emph{Memorization is not learning.}

\hfill --- Richard Phillips Feynman
\end{quote}

Avendo insegnato molte volte in passato un corso introduttivo di analisi dei dati ho notato nel corso degli anni che gli studenti con l'atteggiamento mentale che descriverò qui sotto generalmente ottengono ottimi risultati. Alcuni studenti sviluppano naturalmente questo approccio allo studio, ma altri hanno bisogno di fare uno sforzo per maturarlo. Fornisco qui sotto una breve descrizione del ``metodo di studio'\,' che, nella mia esperienza, è il più efficace per affrontare le richieste di questo insegnamento \autocite{burger20125}.

\begin{itemize}
\tightlist
\item
  Dedicate un tempo sufficiente al materiale di base, apparentemente facile; assicuratevi di averlo capito bene. Cercate le lacune nella vostra comprensione. Leggere presentazioni diverse dello stesso materiale (in libri o articoli diversi) può fornire nuove intuizioni.
\end{itemize}

\begin{itemize}
\item
  Gli errori che facciamo sono i nostri migliori maestri. Istintivamente cerchiamo di dimenticare subito i nostri errori. Ma il miglior modo di imparare è apprendere dagli errori che commettiamo. In questo senso, una soluzione corretta è meno utile di una soluzione sbagliata. Quando commettiamo un errore questo ci fornisce un'informazione importante: ci fa capire qual è il materiale di studio sul quale dobbiamo ritornare e che dobbiamo capire meglio.
\item
  C'è ovviamente un aspetto ``psicologico'' nello studio. Quando un esercizio o problema ci sembra incomprensibile, la cosa migliore da fare è dire: ``mi arrendo'', ``non ho idea di cosa fare!''. Questo ci rilassa: ci siamo già arresi, quindi non abbiamo niente da perdere, non dobbiamo più preoccuparci. Ma non dobbiamo fermarci qui. Le cose ``migliori'' che faccio (se ci sono) le faccio quando non ho voglia di lavorare. Alle volte, quando c'è qualcosa che non so fare e non ho idea di come affontare, mi dico: ``oggi non ho proprio voglia di fare fatica'', non ho voglia di mettermi nello stato mentale per cui ``in 10 minuti devo risolvere il problema perché dopo devo fare altre cose''. Però ho voglia di \emph{divertirmi} con quel problema e allora mi dedico a qualche aspetto ``marginale'' del problema, che so come affrontare, oppure considero l'aspetto più difficile del problema, quello che non so come risolvere, ma invece di cercare di risolverlo, guardo come altre persone hanno affrontato problemi simili, opppure lo stesso problema in un altro contesto. Non mi pongo l'obiettivo ``risolvi il problema in 10 minuti'', ma invece quello di farmi un'idea ``generale'' del problema, o quello di capire un caso più specifico e più semplice del problema. Senza nessuna pressione. Infatti, in quel momento ho deciso di non lavorare (ovvero, di non fare fatica). Va benissimo se ``parto per la tangente'', ovvero se mi metto a leggere del materiale che sembra avere poco a che fare con il problema centrale (le nostre intuizioni e la nostra curiosità solitamente ci indirizzano sulla strada giusta). Quando faccio così, molto spesso trovo la soluzione del problema che mi ero posto e, paradossalmente, la trovo in un tempo minore di quello che, in precedenza, avevo dedicato a ``lavorare'' al problema. Allora perché non faccio sempre così? C'è ovviamente l'aspetto dei ``10 minuti'' che non è sempre facile da dimenticare. Sotto pressione, possiamo solo agire in maniera automatica, ovvero possiamo solo applicare qualcosa che già sappiamo fare. Ma se dobbiamo imparare qualcosa di nuovo, la pressione è un impedimento.
\item
  È utile farsi da soli delle domande sugli argomenti trattati, senza limitarsi a cercare di risolvere gli esercizi che vengono assegnati. Quando studio qualcosa mi viene in mente: ``se questo è vero, allora deve succedere quest'altra cosa''. Allora verifico se questo è vero, di solito con una simulazione. Se i risultati della simulazione sono quelli che mi aspetto, allora vuol dire che ho capito. Se i risultati sono diversi da quelli che mi aspettavo, allora mi rendo conto di non avere capito e ritorno indietro a studiare con più attenzione la teoria che pensavo di avere capito -- e ovviamente mi rendo conto che c'era un aspetto che avevo frainteso. Questo tipo di verifica è qualcosa che dobbiamo fare da soli, in prima persona: nessun altro può fare questo al posto nostro.
\item
  Non aspettatevi di capire tutto la prima volta che incontrate un argomento nuovo.\footnote{Ricordatevi inoltre che gli individui tendono a sottostimare la propria capacità di apprendere \autocite{horn2021underestimating}.} È utile farsi una nota mentalmente delle lacune nella vostra comprensione e tornare su di esse in seguito per carcare di colmarle. L'atteggiamento naturale, quando non capiamo i dettagli di qualcosa, è quello di pensare: ``non importa, ho capito in maniera approssimativa questo punto, non devo preoccuparmi del resto''. Ma in realtà non è vero: se la nostra comprensione è superficiale, quando il problema verrà presentato in una nuova forma, non riusciremo a risolverlo. Per cui i dubbi che ci vengono quando studiamo qualcosa sono il nostro alleato più prezioso: ci dicono esattamente quali sono gli aspetti che dobbiamo approfondire per potere migliorare la nostra preparazione.
\item
  È utile sviluppare una visione d'insieme degli argomenti trattati, capire l'obiettivo generale che si vuole raggiungere e avere chiaro il contributo che i vari pezzi di informazione forniscono al raggiungimento di tale obiettivo. Questa organizzazione mentale del materiale di studio facilita la comprensione. È estremamente utile creare degli schemi di ciò che si sta studiando. Non aspettate che sia io a fornirvi un riepilogo di ciò che dovete imparare: sviluppate da soli tali schemi e tali riassunti.
\item
  Tutti noi dobbiamo imparare l'arte di trovare le informazioni, non solo nel caso di questo insegnamento. Quando vi trovate di fronte a qualcosa che non capite, o ottenete un oscuro messaggio di errore da un software, ricordatevi: ``Google is your friend''.
\end{itemize}

\bigskip

Corrado Caudek

\mainmatter

\hypertarget{chapter-distr-coniugate}{%
\chapter{Distribuzioni a priori coniugate}\label{chapter-distr-coniugate}}

\begin{chapterintro}
Obiettivo di questo Capitolo è fornire un esempio di derivazione della distribuzione a posteriori scegliendo quale distribuzione a priori una distribuzione coniugata. Esamineremo qui il modello Beta-Binomiale.

\end{chapterintro}

\hypertarget{il-denominatore-bayesiano}{%
\section{Il denominatore bayesiano}\label{il-denominatore-bayesiano}}

In un problema bayesiano i dati \(y\) provengono da una distribuzione \(p(y \mid \theta)\) e al parametro \(\theta\) viene assegnata una distribuzione a priori \(p(\theta)\). La scelta della distribuzione a priori ha importanti conseguenze di tipo computazionale. Infatti, a meno di non utilizzare particolari forme analitiche, risulta impossibile ottenere espressioni esplicite per la distribuzione a posteriori. Ciò dipende dall'espressione a denominatore della formula di Bayes
\begin{equation}
p(\theta \mid y) = \frac{p(\theta) p(y \mid \theta)}{\int p(\theta) p(y \mid \theta) \,\operatorname {d}\! \theta} \notag
\end{equation}
il cui calcolo non è eseguibile in modo analitico in forma chiusa. Per non incorrere in problemi nel calcolo della distribuzione a posteriori vengono usate le distribuzioni provenienti da famiglie coniugate.

\begin{definition}
\protect\hypertarget{def:def-conj-fam}{}{\label{def:def-conj-fam} }Una distribuzione di probabilità a priori \(p(\theta)\) si dice \emph{coniugata} al modello usato se la distribuzione a priori e la distribuzione a posteriori hanno la stessa forma funzionale. Dunque, le due distribuzioni differiscono solo per il valore dei parametri.
\end{definition}

È possibile ottenere la distribuzione posteriore per via analitica solo per alcune specifiche combinazioni di distribuzione a priori e verosimiglianza, ma questo limita considerevolmente la flessibilità della modellizzazione.\footnote{Per questa ragione, la strada principale che viene seguita nella modellistica bayesiana è quella che porta a determinare la distribuzione a posteriori non per via analitica, ma bensì mediante metodi numerici. La simulazione fornisce dunque la strategia generale del calcolo bayesiano. A questo fine vengono usati i metodi di campionamento detti Monte-Carlo Markov-Chain (MCMC). Tali metodi costituiscono una potente e praticabile alternativa per la costruzione della distribuzione a posteriori per modelli complessi e consentono di decidere quali distribuzioni a priori e quali distribuzioni di verosimiglianza usare sulla base di considerazioni teoriche soltanto, senza dovere preoccuparsi di altri vincoli. Dato che è basata su metodi computazionalmente intensivi, la stima numerica della funzione a posteriori può essere svolta soltanto mediante software. In anni recenti i metodi bayesiani di analisi dei dati sono diventati sempre più popolari proprio perché la potenza di calcolo necessaria per svolgere tali calcoli è ora alla portata di tutti. Questo non era vero solo pochi decenni fa.}

\hypertarget{chapter-distr-priori-coniugate}{%
\section{Il modello Beta-Binomiale}\label{chapter-distr-priori-coniugate}}

Per fare un esempio concreto, consideriamo nuovamente i dati di \textcite{zetschefuture2019}: nel campione di 30 partecipanti clinici le aspettative future di 23 partecipanti risultano distorte negativamente e quelle di 7 partecipanti risultano distorte positivamente. Nel seguito, indicheremo con \(\theta\) la probabilità che le aspettative di un paziente clinico siano distorte negativamente. Ci poniamo il problema di ottenere una stima a posteriori di \(\theta\) avendo osservato 23 ``successi'' in 30 prove.

I dati osservati (\(y = 23\)) possono essere considerati la manifestazione di una variabile casuale Bernoulliana. In tali circostanze, esiste una famiglia di distribuzioni che, qualora venga scelta per la distribuzione a priori, fa sì che la distribuzione a posteriori abbia la stessa forma funzionale della distribuzione a priori. Questo consente una soluzione analitica dell'integrale che compare a denominatore nella formula di Bayes. Nel caso presente, la famiglia di distribuzioni che ha questa proprietà è la distribuzione Beta.

\hypertarget{parametri-della-distribuzione-beta}{%
\subsection{Parametri della distribuzione Beta}\label{parametri-della-distribuzione-beta}}

È possibile esprimere diverse credenze iniziali rispetto a \(\theta\) mediante la distribuzione Beta. Ad esempio, la scelta di una \(\mbox{Beta}(\alpha = 4, \beta = 4)\) quale distribuzione a priori per il parametro \(\theta\) corrisponde alla credenza a priori che associa all'evento ``presenza di una aspettativa futura distorta negativamente'' una grande incertezza: il valore 0.5 è il valore di \(\theta\) più plausibile, ma anche gli altri valori del parametro (tranne gli estremi) sono ritenuti piuttosto plausibili. Questa distribuzione a priori esprime la credenza che sia egualmente probabile per un'aspettativa futura essere distorta negativamente o positivamente.

\begin{Shaded}
\begin{Highlighting}[]
\FunctionTok{library}\NormalTok{(}\StringTok{"bayesrules"}\NormalTok{)}
\FunctionTok{plot\_beta}\NormalTok{(}\AttributeTok{alpha =} \DecValTok{4}\NormalTok{, }\AttributeTok{beta =} \DecValTok{4}\NormalTok{, }\AttributeTok{mean =} \ConstantTok{TRUE}\NormalTok{, }\AttributeTok{mode =} \ConstantTok{TRUE}\NormalTok{)}
\end{Highlighting}
\end{Shaded}

\begin{center}\includegraphics[width=0.8\linewidth]{026_conjugate_families_files/figure-latex/unnamed-chunk-1-1} \end{center}

Possiamo quantificare la nostra incertezza calcolando, con un grado di fiducia del 95\%, la regione nella quale, in base a tale credenza a priori, si trova il valore del parametro. Per ottenere tale intervallo di credibilità a priori, usiamo la funzione \texttt{qbeta()} di \(\R\). In \texttt{qbeta()} i parametri \(\alpha\) e \(\beta\) sono chiamati \texttt{shape1} e \texttt{shape2}:

\begin{Shaded}
\begin{Highlighting}[]
\FunctionTok{qbeta}\NormalTok{(}\FunctionTok{c}\NormalTok{(}\FloatTok{0.025}\NormalTok{, }\FloatTok{0.975}\NormalTok{), }\AttributeTok{shape1 =} \DecValTok{4}\NormalTok{, }\AttributeTok{shape2 =} \DecValTok{4}\NormalTok{)}
\CommentTok{\#\textgreater{} [1] 0.184 0.816}
\end{Highlighting}
\end{Shaded}

Se poniamo \(\alpha=10\) e \(\beta=10\), questo corrisponde ad una credenza a priori che sia egualmente probabile per un'aspettativa futura essere distorta negativamente o positivamente,

\begin{Shaded}
\begin{Highlighting}[]
\FunctionTok{plot\_beta}\NormalTok{(}\AttributeTok{alpha =} \DecValTok{10}\NormalTok{, }\AttributeTok{beta =} \DecValTok{10}\NormalTok{, }\AttributeTok{mean =} \ConstantTok{TRUE}\NormalTok{, }\AttributeTok{mode =} \ConstantTok{TRUE}\NormalTok{)}
\end{Highlighting}
\end{Shaded}

\begin{center}\includegraphics[width=0.8\linewidth]{026_conjugate_families_files/figure-latex/unnamed-chunk-3-1} \end{center}

\noindent
ma ora la nostra certezza a priori sul valore del parametro è maggiore, come indicato dall'intervallo al 95\%:

\begin{Shaded}
\begin{Highlighting}[]
\FunctionTok{qbeta}\NormalTok{(}\FunctionTok{c}\NormalTok{(}\FloatTok{0.025}\NormalTok{, }\FloatTok{0.975}\NormalTok{), }\AttributeTok{shape1 =} \DecValTok{10}\NormalTok{, }\AttributeTok{shape2 =} \DecValTok{10}\NormalTok{)}
\CommentTok{\#\textgreater{} [1] 0.289 0.711}
\end{Highlighting}
\end{Shaded}

Quale distribuzione a priori dobbiamo scegliere? In un problema concreto di analisi dei dati, la scelta della distribuzione a priori dipende dalle credenze a priori che vogliamo includere nell'analisi dei dati. Se non abbiamo alcuna informazione a priori, potremmo usare \(\alpha=1\) e \(\beta=1\), che produce una distribuzione a priori uniforme. Ma l'uso di distribuzioni a priori uniformi è sconsigliato per vari motivi, inclusa l'instabilità numerica della stima dei parametri. È meglio invece usare una distribuzione a priori poco informativa, come \(\mbox{Beta}(2, 2)\).

Nella discussione successiva, solo per fare un esempio, useremo quale distribuzione a priori una \(\mbox{Beta}(2, 10)\), ovvero:
\[
p(\theta) = \frac{\Gamma(12)}{\Gamma(2)\Gamma(10)}\theta^{2-1} (1-\theta)^{10-1}.
\]

\begin{Shaded}
\begin{Highlighting}[]
\FunctionTok{plot\_beta}\NormalTok{(}\AttributeTok{alpha =} \DecValTok{2}\NormalTok{, }\AttributeTok{beta =} \DecValTok{10}\NormalTok{, }\AttributeTok{mean =} \ConstantTok{TRUE}\NormalTok{, }\AttributeTok{mode =} \ConstantTok{TRUE}\NormalTok{)}
\end{Highlighting}
\end{Shaded}

\begin{center}\includegraphics[width=0.8\linewidth]{026_conjugate_families_files/figure-latex/unnamed-chunk-5-1} \end{center}

\noindent
La \(\mbox{Beta}(2, 10)\) esprime la credenza che \(\theta < 0.5\), con il valore più plausibile pari a cicrca 0.1.

\hypertarget{la-specificazione-della-distribuzione-a-posteriori}{%
\subsection{La specificazione della distribuzione a posteriori}\label{la-specificazione-della-distribuzione-a-posteriori}}

Una volta scelta una distribuzione a priori di tipo Beta, i cui parametri rispecchiano le nostre credenze iniziali su \(\theta\), la distribuzione a posteriori viene specificata dalla formula di Bayes:
\[
\text{distribuzione a posteriori} = \frac{\text{verosimiglianza}\cdot\text{distribuzione a priori}}{\text{verosimiglianza marginale}}.
\]
Nel caso presente abbiamo
\[
p(\theta \mid n=30, y=23) = \frac{\Big[\binom{30}{23}\theta^{23}(1-\theta)^{30-23}\Big]\Big[\frac{\Gamma(12)}{\Gamma(2)\Gamma(10)}\theta^{2-1} (1-\theta)^{10-1}\Big]}{p(y = 23)},
\]
laddove \(p(y = 23)\), ovvero la verosimiglianza marginale, è una costante di normalizzazione che fa sì che l'area sottesa alla densità a posteriori sia unitaria.

Riscriviamo ora l'equazione precedente in termini generali
\[
p(\theta \mid n, y) = \frac{\Big[\binom{n}{y}\theta^{y}(1-\theta)^{n-y}\Big]\Big[\frac{\Gamma(a+b)}{\Gamma(a)\Gamma(b)}\theta^{a-1} (1-\theta)^{b-1}\Big]}{p(y)}
\]
\noindent
e raccogliendo tutte le costanti otteniamo:
\[
p(\theta \mid n, y) =\left[\frac{\binom{n}{y}\frac{\Gamma(a+b)}{\Gamma(a)\Gamma(b)}}{p(y)}\right] \theta^{y}(1-\theta)^{n-y}\theta^{a-1} (1-\theta)^{b-1}.
\]
\noindent
Se ignoriamo il termine costante all'interno della parentesi quadra
\begin{align}
p(\theta \mid n, y) &\propto \theta^{y}(1-\theta)^{n-y}\theta^{a-1} (1-\theta)^{b-1},\notag\\
&\propto \theta^{a+y-1}(1-\theta)^{b+n-y-1},\notag
\end{align}
\noindent
il termine di destra dell'equazione precedente identifica il \emph{kernel} della distribuzione a posteriori e corrisponde ad una Beta \emph{non normalizzata} di parametri \(a + y\) e \(b + n - y\).

Per ottenere una distribuzione di densità, dobbiamo aggiungere una costante di normalizzazione al kernel della distribuzione a posteriori. In base alla definizione della distribuzione Beta, ed essendo \(a' = a+y\) e \(b' = b+n-y\), tale costante di normalizzazione sarà uguale a
\[
\frac{\Gamma(a'+b')}{\Gamma(a')\Gamma(b')} = \frac{\Gamma(a+b+n)}{\Gamma(a+y)\Gamma(b+n-y)}.
\]
\noindent
In altri termini, la distribuzione a posteriori diventa una \(\mbox{Beta}(a+y, b+n-y)\):
\[
\mbox{Beta}(a+y, b+n-y) = \frac{\Gamma(a+b+n)}{\Gamma(a+y)\Gamma(b+n-y)} \theta^{a+y-1}(1-\theta)^{b+n-y-1}.
\]

Possiamo concludere dicendo che siamo partiti da una verosimiglianza \(\Bin(n = 30, y = 23 \mid \theta)\). Moltiplicando la verosimiglianza per la distribuzione a priori \(\theta \sim \mbox{Beta}(2, 10)\), abbiamo ottenuto la distribuzione a posteriori \(p(\theta \mid n, y) \sim \mbox{Beta}(25, 17)\). Questo è un esempio di analisi coniugata: la distribuzione a posteriori del parametro ha la stessa forma funzionale della distribuzione a priori. La presente combinazione di verosimiglianza e distribuzione a priori è chiamata caso coniugato \emph{Beta-Binomiale} ed è descritto dal seguente teorema.

\begin{theorem}
\protect\hypertarget{thm:beta-binom}{}\label{thm:beta-binom}Sia data la funzione di verosimiglianza \(\Bin(n, y \mid \theta)\) e sia \(\mbox{Beta}(\alpha, \beta)\) una distribuzione a priori. In tali circostanze, la distribuzione a posteriori del parametro \(\theta\) sarà una distribuzione \(\mbox{Beta}(\alpha + y, \beta + n - y)\).
\end{theorem}

È facile calcolare il valore atteso a posteriori di \(\theta\). Essendo \(\E[\mbox{Beta}(\alpha, \beta)] = \frac{\alpha}{\alpha + \beta}\), il risultato cercato diventa
\begin{equation}
\E_{\text{post}} [\mathrm{Beta}(\alpha + y, \beta + n - y)] = \frac{\alpha + y}{\alpha + \beta +n}.
\label{eq:ev-post-beta-bin-1}
\end{equation}

\begin{guidedpractice}
Usando le funzione \(\R\) \texttt{plot\_beta\_binomial()} e \texttt{plot\_beta\_binomial()} del pacchetto \texttt{bayesrules}, si rappresenti in maniera grafica e si descriva in forma numerica l'aggiornamento bayesiano Beta-Binomiale per i dati di \textcite{zetschefuture2019}.

\end{guidedpractice}

Per i dati in discussione, abbiamo:

\begin{Shaded}
\begin{Highlighting}[]
\NormalTok{bayesrules}\SpecialCharTok{::}\FunctionTok{plot\_beta\_binomial}\NormalTok{(}
  \AttributeTok{alpha =} \DecValTok{2}\NormalTok{, }\AttributeTok{beta =} \DecValTok{10}\NormalTok{, }\AttributeTok{y =} \DecValTok{23}\NormalTok{, }\AttributeTok{n =} \DecValTok{30}
\NormalTok{)}
\end{Highlighting}
\end{Shaded}

\begin{center}\includegraphics[width=0.8\linewidth]{026_conjugate_families_files/figure-latex/unnamed-chunk-6-1} \end{center}

\noindent
Un sommario delle distribuzioni a priori e a posteriori si ottiene usando la funzione \texttt{summarize\_beta\_binomial()}:

\begin{Shaded}
\begin{Highlighting}[]
\NormalTok{bayesrules}\SpecialCharTok{:::}\FunctionTok{summarize\_beta\_binomial}\NormalTok{(}
  \AttributeTok{alpha =} \DecValTok{2}\NormalTok{, }\AttributeTok{beta =} \DecValTok{10}\NormalTok{, }\AttributeTok{y =} \DecValTok{23}\NormalTok{, }\AttributeTok{n =} \DecValTok{30}
\NormalTok{)}
\CommentTok{\#\textgreater{}       model alpha beta  mean mode    var     sd}
\CommentTok{\#\textgreater{} 1     prior     2   10 0.167  0.1 0.0107 0.1034}
\CommentTok{\#\textgreater{} 2 posterior    25   17 0.595  0.6 0.0056 0.0749}
\end{Highlighting}
\end{Shaded}

\begin{guidedpractice}
Per i dati di \textcite{zetschefuture2019}, si trovino la media, la moda, la deviazione standard della distribuzione a posteriori di \(\theta\). Si trovi inoltre l'intervallo di credibilità a posteriori del 95\% per il parametro \(\theta\).

\end{guidedpractice}

Usando la \ref{thm:beta-binom}, possiamo ottenere l'intervallo di credibilità a posteriori del 95\% per il parametro \(\theta\) come segue:

\begin{Shaded}
\begin{Highlighting}[]
\FunctionTok{qbeta}\NormalTok{(}\FunctionTok{c}\NormalTok{(}\FloatTok{0.025}\NormalTok{, }\FloatTok{0.975}\NormalTok{), }\AttributeTok{shape1 =} \DecValTok{25}\NormalTok{, }\AttributeTok{shape2 =} \DecValTok{17}\NormalTok{)}
\CommentTok{\#\textgreater{} [1] 0.445 0.737}
\end{Highlighting}
\end{Shaded}

\noindent
La media della distribuzione a posteriori è

\begin{Shaded}
\begin{Highlighting}[]
\DecValTok{25} \SpecialCharTok{/}\NormalTok{ (}\DecValTok{25} \SpecialCharTok{+} \DecValTok{17}\NormalTok{)}
\CommentTok{\#\textgreater{} [1] 0.595}
\end{Highlighting}
\end{Shaded}

\noindent
La moda della distribuzione a posteriori è

\begin{Shaded}
\begin{Highlighting}[]
\NormalTok{(}\DecValTok{25} \SpecialCharTok{{-}} \DecValTok{1}\NormalTok{) }\SpecialCharTok{/}\NormalTok{ (}\DecValTok{25} \SpecialCharTok{+} \DecValTok{17} \SpecialCharTok{{-}} \DecValTok{2}\NormalTok{)}
\CommentTok{\#\textgreater{} [1] 0.6}
\end{Highlighting}
\end{Shaded}

\noindent
La deviazione standard della distribuzione a priori è

\begin{Shaded}
\begin{Highlighting}[]
\FunctionTok{sqrt}\NormalTok{((}\DecValTok{25} \SpecialCharTok{*} \DecValTok{17}\NormalTok{) }\SpecialCharTok{/}\NormalTok{ ((}\DecValTok{25} \SpecialCharTok{+} \DecValTok{17}\NormalTok{)}\SpecialCharTok{\^{}}\DecValTok{2} \SpecialCharTok{*}\NormalTok{ (}\DecValTok{25} \SpecialCharTok{+} \DecValTok{17} \SpecialCharTok{+} \DecValTok{1}\NormalTok{)))}
\CommentTok{\#\textgreater{} [1] 0.0749}
\end{Highlighting}
\end{Shaded}

\begin{guidedpractice}
Si trovino i parametri e le proprietà della distribuzione a posteriori del parametro \(\theta\) per i dati dell'esempio relativo alla ricerca di Stanley Milgram discussa da \textcite{Johnson2022bayesrules}.

\end{guidedpractice}

Nel 1963, Stanley Milgram presentò una ricerca sulla propensione delle persone a obbedire agli ordini di figure di autorità, anche quando tali ordini possono danneggiare altre persone \autocite{milgram1963behavioral}. Nell'articolo, Milgram descrive lo studio come

\begin{quote}
consist{[}ing{]} of ordering a naive subject to administer electric shock to a victim. A simulated shock generator is used, with 30 clearly marked voltage levels that range from IS to 450 volts. The instrument bears verbal designations that range from Slight Shock to Danger: Severe Shock. The responses of the victim, who is a trained confederate of the experimenter, are standardized. The orders to administer shocks are given to the naive subject in the context of a `learning experiment' ostensibly set up to study the effects of punishment on memory. As the experiment proceeds the naive subject is commanded to administer increasingly more intense shocks to the victim, even to the point of reaching the level marked Danger: Severe Shock.
\end{quote}

\noindent
All'insaputa del partecipante, gli shock elettrici erano falsi e l'attore stava solo fingendo di provare il dolore dello shock.

\textcite{Johnson2022bayesrules} fanno inferenza sui risultati dello studio di Milgram mediante il modello Beta-Binomiale. Il parametro di interesse è \(\theta\), la probabiltà che una persona obbedisca all'autorità (in questo caso, somministrando lo shock più severo), anche se ciò significa recare danno ad altri. \textcite{Johnson2022bayesrules} ipotizzano che, prima di raccogliere dati, le credenze di Milgram relative a \(\theta\) possano essere rappresentate mediante una \(\mbox{Beta}(1, 10)\). Sia \(y = 26\) il numero di soggetti che, sui 40 partecipanti allo studio, aveva accettato di infliggere lo shock più severo. Assumendo che ogni partecipante si comporti indipendentemente dagli altri, possiamo modellare la dipendenza di \(y\) da \(\theta\) usando la distribuzione binomiale. Giungiamo dunque al seguente modello bayesiano Beta-Binomiale:
\begin{align}
y \mid \theta & \sim \Bin(n = 40, \theta) \notag\\
\theta & \sim \text{Beta}(1, 10) \; . \notag
\end{align}
Usando le funzioni di \texttt{bayesrules} possiamo facilmente calcolare i parametri e le proprietà della distribuzione a posteriori:

\begin{Shaded}
\begin{Highlighting}[]
\NormalTok{bayesrules}\SpecialCharTok{:::}\FunctionTok{summarize\_beta\_binomial}\NormalTok{(}
  \AttributeTok{alpha =} \DecValTok{1}\NormalTok{, }\AttributeTok{beta =} \DecValTok{10}\NormalTok{, }\AttributeTok{y =} \DecValTok{26}\NormalTok{, }\AttributeTok{n =} \DecValTok{40}
\NormalTok{)}
\CommentTok{\#\textgreater{}       model alpha beta   mean  mode     var     sd}
\CommentTok{\#\textgreater{} 1     prior     1   10 0.0909 0.000 0.00689 0.0830}
\CommentTok{\#\textgreater{} 2 posterior    27   24 0.5294 0.531 0.00479 0.0692}
\end{Highlighting}
\end{Shaded}

\noindent
Il processo di aggiornamento bayesiano è descritto dalla figura seguente:

\begin{Shaded}
\begin{Highlighting}[]
\NormalTok{bayesrules}\SpecialCharTok{:::}\FunctionTok{plot\_beta\_binomial}\NormalTok{(}
  \AttributeTok{alpha =} \DecValTok{1}\NormalTok{, }\AttributeTok{beta =} \DecValTok{10}\NormalTok{, }\AttributeTok{y =} \DecValTok{26}\NormalTok{, }\AttributeTok{n =} \DecValTok{40}
\NormalTok{)}
\end{Highlighting}
\end{Shaded}

\begin{center}\includegraphics[width=0.8\linewidth]{026_conjugate_families_files/figure-latex/unnamed-chunk-13-1} \end{center}

\hypertarget{principali-distribuzioni-coniugate}{%
\section{Principali distribuzioni coniugate}\label{principali-distribuzioni-coniugate}}

Esistono molte altre combinazioni simili di verosimiglianza e distribuzione a priori le quali producono una distribuzione a posteriori che ha la stessa densità della distribuzione a priori. Sono elencate qui sotto le più note coniugazioni tra modelli statistici e distribuzioni a priori.

\begin{itemize}
\item
  Per il modello Normale-Normale \(\mathcal{N}(\mu, \sigma^2_0)\), la distribizione iniziale è \(\mathcal{N}(\mu_0, \tau^2)\) e la distribuzione finale è \(\mathcal{N}\left(\frac{\mu_0\sigma^2 + \bar{y}n\tau^2}{\sigma^2 + n\tau^2}, \frac{\sigma^2\tau^2}{\sigma^2 + n\tau^2} \right)\).
\item
  Per il modello Poisson-gamma \(\text{Po}(\theta)\), la distribizione iniziale è \(\Gamma(\lambda, \delta)\) e la distribuzione finale è \(\Gamma(\lambda + n \bar{y}, \delta +n)\).
\item
  Per il modello esponenziale \(\text{Exp}(\theta)\), la distribizione iniziale è \(\Gamma(\lambda, \delta)\) e la distribuzione finale è \(\Gamma(\lambda + n, \delta +n\bar{y})\).
\item
  Per il modello uniforme-Pareto \(\text{U}(0, \theta)\), la distribizione iniziale è \(\mbox{Pa}(\alpha, \varepsilon)\) e la distribuzione finale è \(\mbox{Pa}(\alpha + n, \max(y_{(n)}, \varepsilon))\).
\end{itemize}

\hypertarget{considerazioni-conclusive}{%
\section*{Considerazioni conclusive}\label{considerazioni-conclusive}}
\addcontentsline{toc}{section}{Considerazioni conclusive}

Lo scopo di questa discussione è stato quello di mostrare come sia possibile combinare le nostre conoscenze a priori (espresse nei termini di una densità di probabilità) con le evidenze fornite dai dati (espresse nei termini della funzione di verosimiglianza), così da determinare, mediante il teorema di Bayes, una distribuzione a posteriori, la quale condensa l'incertezza che abbiamo sul parametro \(\theta\). Per illustrare tale problema, abbiamo considerato una situazione nella quale \(\theta\) corrisponde alla probabilità di successo in una sequenza di prove Bernoulliane. Abbiamo visto come, in queste circostanze, sia ragionevole esprimere le nostre credenze a priori mediante la densità Beta, con opportuni parametri. L'inferenza rispetto ad una proporzione rappresenta un caso particolare, ovvero un caso nel quale la distribuzione a priori è Beta e la verosimiglianza è Binomiale. In tali circostanze, la distribuzione a posteriori diventa una distribuzione Beta -- questo è il cosiddetto modello Beta-Binomiale. Dato che utilizza una distribuzione a priori coniugata, dunque, il modello Beta-Binomiale rende possibile la determinazione analitica dei parametri della distribuzione a posteriori.


% Bibliography
%%%%%%%%%%%%%%%%%%%%%%%%%%%%%%%%%%%%%%%%%%%%%%%%%%%%%%%%%%

\backmatter
\SmallMargins

\printbibliography
\onecolumn


% Tables (of tables, of figures)
%%%%%%%%%%%%%%%%%%%%%%%%%%%%%%%%%%%%%%%%%%%%%%%%%%%%%%%%%%


\cleardoublepage
\LargeMargins
\listoffigures


% After-body (LaTeX code inclusion)
%%%%%%%%%%%%%%%%%%%%%%%%%%%%%%%%%%%%%%%%%%%%%%%%%%%%%%%%%%




% Back cover
%%%%%%%%%%%%%%%%%%%%%%%%%%%%%%%%%%%%%%%%%%%%%%%%%%%%%%%%%%%

% Even page, small margins, no running head, no page number.
\evenpage
\SmallMargins
\thispagestyle{empty}

\begin{normalsize}

\begin{description}

\selectlanguage{italian}
\item[Abstract]
This document contains the material of the lessons of Psicometria B000286 (2021/2022) aimed at students of the first year of the Degree Course in Psychological Sciences and Techniques of the University of Florence, Italy.
\item[Keywords]
Data science, Bayesian statistics.
~\\

\end{description}

\end{normalsize}


\end{document}
